\makeatletter

%%%%%%%%%%%%%%%%%%%%%%%%%%%%%%%%%%%%%%%%%%%%%%%%%%%%%%%%%%%%%%%%%%%%%%%%%%%%%%%%
% ..:: Colors ::..
\definecolor{beamerDarkBlue}{HTML}{26415d}
\definecolor{beamerBlue}{HTML}{356397}
\definecolor{beamerGreen}{HTML}{5da9a1}
\definecolor{beamerYellow}{HTML}{f1d46a}
\definecolor{beamerRed}{HTML}{db6245}
\definecolor{beamerBlack}{HTML}{000000}
\definecolor{beamerGrey}{HTML}{acacac}
%%%%%%%%%%%%%%%%%%%%%%%%%%%%%%%%%%%%%%%%%%%%%%%%%%%%%%%%%%%%%%%%%%%%%%%%%%%%%%%%

%%%%%%%%%%%%%%%%%%%%%%%%%%%%%%%%%%%%%%%%%%%%%%%%%%%%%%%%%%%%%%%%%%%%%%%%%%%%%%%%
% ..:: Font ::..
% Font size
\usepackage{scrextend}
\changefontsizes{8pt}

% Check and cross marks
\usepackage{pifont}
\newcommand{\cmark}{\ding{51}}%
\newcommand{\xmark}{\ding{55}}%
\newcommand{\goodmark}{{\color{beamerGreen}\cmark}}
\newcommand{\badmark}{{\color{beamerRed}\xmark}}
%%%%%%%%%%%%%%%%%%%%%%%%%%%%%%%%%%%%%%%%%%%%%%%%%%%%%%%%%%%%%%%%%%%%%%%%%%%%%%%%

%%%%%%%%%%%%%%%%%%%%%%%%%%%%%%%%%%%%%%%%%%%%%%%%%%%%%%%%%%%%%%%%%%%%%%%%%%%%%%%%
% ..:: Text highlighting ::..
% From answer: https://tex.stackexchange.com/a/7545/90535
\newlength{\hl@depth@ref}
\settodepth{\hl@depth@ref}{g}
\newlength{\hl@height@ref}
\settoheight{\hl@height@ref}{T}
\newlength{\hl@width}%
\newcommand{\beamerhighlight}[2][beamerYellow]%
{%
  \settowidth{\hl@width}{#2}%
  \colorbox{#1}%
  {%      
    \raisebox{-\hl@depth@ref}%
    {%
      \parbox[b][\hl@height@ref+\hl@depth@ref][c]{\hl@width}{\centering#2}%
    }%
  }%
}
%%%%%%%%%%%%%%%%%%%%%%%%%%%%%%%%%%%%%%%%%%%%%%%%%%%%%%%%%%%%%%%%%%%%%%%%%%%%%%%% 

%%%%%%%%%%%%%%%%%%%%%%%%%%%%%%%%%%%%%%%%%%%%%%%%%%%%%%%%%%%%%%%%%%%%%%%%%%%%%%%%
% ..:: TikZ elements ::..
% Hyperlinkable nodes
% from: https://tex.stackexchange.com/a/36111/90535
\tikzset{
  hyperlink node/.style={
    alias=sourcenode,
    append after command={
      let     \p1 = (sourcenode.north west),
      \p2=(sourcenode.south east),
      \n1={\x2-\x1},
      \n2={\y1-\y2} in
      node [inner sep=0pt, outer sep=0pt,anchor=north west,at=(\p1)]
      {\hyperlink{#1}{\XeTeXLinkBox{\phantom{\rule{\n1}{\n2}}}}}
    }
  }
}
%%%%%%%%%%%%%%%%%%%%%%%%%%%%%%%%%%%%%%%%%%%%%%%%%%%%%%%%%%%%%%%%%%%%%%%%%%%%%%%%

%%%%%%%%%%%%%%%%%%%%%%%%%%%%%%%%%%%%%%%%%%%%%%%%%%%%%%%%%%%%%%%%%%%%%%%%%%%%%%%%
% ..:: Footer: progress bar and sections list ::..
\setbeamertemplate{navigation symbols}{}
% >> Progress variables <<
\newcommand{\slidefooter}[1]{
  \def\slidefooter@text{#1}
}
\def\footer@text@xshift{1mm}
\def\pagecounter@top{}
% >> Section list variables <<
\def\short@section{}
\def\beamer@sections{}
\def\saved@beamer@sections{}
\def\beamer@appendices{}
\def\saved@beamer@appendices{}
\def\appendix@started{false}
\def\appendix@start@slide{}
\def\saved@appendix@start@slide{-1}
\newcommand{\beamersection}[2][]{
  % Add a new section to the slides
  \pykzexec{options='#1'.split(',')}
  \pykzmathinline[is@appendix]{'true' if 'appendix' in options else 'false'}
  \pykzmathinline[short@section]{''.join(s for s in options if s!='appendix')}
  \ifthenelse{\equal{\short@section}{}}{
    \xdef\short@section{#2}
  }{}
  \ifthenelse{\equal{\is@appendix}{false} \AND \equal{\appendix@started}{true}}{
    \PackageError{beamer}{Cannot have non-appendix sections after appendix has
      started}{Include the appendix optional argument}
  }{
    \ifthenelse{\equal{\is@appendix}{true} \AND \equal{\appendix@started}{false}}{
      % Sets the state of the presentation into appendix mode
      \xdef\appendix@start@slide{\insertframenumber}
      \xdef\appendix@started{true}
      % Create an appendix demarcation slide
      \begin{frame}
        \tikzset{
          appendix text/.style={}
        }
        \begin{beamertikzpicture}
          \getframecoord{0,0}
          \node[appendix text] at (\framecoord) {%
            \beamerhighlight[beamerDarkBlue]{\color{white}Appendices}};
        \end{beamertikzpicture}
      \end{frame}
    }{}
  }
  \section{#2}
  % Add to list of all sections
  \ifthenelse{\equal{\appendix@started}{false}}{
    \tikzmath{
      \current@slide@number = \insertframenumber;
    }
    \xdef\@variable{beamer@sections}
  }{
    \tikzmath{
      \current@slide@number = int(\insertframenumber-\appendix@start@slide);
    }
    \xdef\@variable{beamer@appendices}
  }
  \ifthenelse{\equal{\csname \@variable\endcsname}{}}{
    \expandafter\xdef\csname \@variable\endcsname{
      \short@section/\current@slide@number}
  }{
    \expandafter\xdef\csname \@variable\endcsname{
      \short@section/\current@slide@number,\csname \@variable\endcsname}
  }
}
%%%%%%%%%%%%%%%%%
% Make the footer
%%%%%%%%%%%%%%%%%
\newcommand{\get@section@color}[1]{\pykzmathinline[this@section@color]{
    section_color_wheel[\section@counter-(\section@counter//n_section_colors)*n_section_colors]}}
\newcommand{\load@aux}{
  % >> Load sections from aux file <<
  \ifthenelse{\equal{\insertframenumber}{1}}{
    % Load sections
    \pykzexec{f = open('../../beamersections.aux','r')}
    \pykzexec{raw_sections = f.read()}
    \pykzexec{sections = raw_sections.split(',')}
    \pykzmathinline[saved@beamer@sections]{raw_sections}
    \pykzexec{f.close()}
    
    % Load appendices
    \pykzexec{f = open('../../beamerappendices.aux','r')}
    \pykzexec{raw_appendices = f.read()}
    \pykzexec{appendices = raw_appendices.split(',')}
    \pykzmathinline[saved@beamer@appendices]{raw_appendices}
    \pykzexec{f.close()}
    \pykzexec{f = open('../../appendixstart.aux','r')}
    \pykzmathinline[saved@appendix@start@slide]{f.read()}
    \pykzexec{f.close()}
  }{}
}
\setbeamertemplate{footline}{
  \load@aux
  
  % >> Write sections to aux file <<
  \ifthenelse{\equal{\insertframenumber}{\inserttotalframenumber}}{
    % Write sections
    \pykzexec{f = open('../../beamersections.aux','w')}
    \pykzexec{print('\beamer@sections',file=f,end='')}
    \pykzexec{f.close()}

    % Write appendices
    \pykzexec{f = open('../../beamerappendices.aux','w')}
    \pykzexec{print('\beamer@appendices',file=f,end='')}
    \pykzexec{f.close()}
    \pykzexec{f = open('../../appendixstart.aux','w')}
    \ifthenelse{\equal{\appendix@start@slide}{}}{
      \pykzexec{print('\inserttotalframenumber',file=f,end='')}
    }{
      \pykzexec{print('\appendix@start@slide',file=f,end='')}
    }
    \pykzexec{f.close()}
  }{}

  % >> TikZ styles <<
  \tikzset{
    every node/.append style= {
      inner sep=0,
      outer sep=0
    },
    progress line/.style = {
      draw=black,
      line width=0.15mm
    },
    appendix line/.style={
      dashed
    },
    progress tip/.style = {
      circle,
      draw=black,
      fill=black,
      minimum size=1mm
    },
    section checkpoint/.style={
      circle,
      draw=##1!70,
      fill=##1!30,
      line width=0.2mm,
      minimum size=1.4mm
    },
    page counter/.style = {
      anchor=south east
    },
    footer text/.style = {
      anchor=south west,
      align=left,
      scale=0.8
    },
    section color bubble/.style = {
      circle,
      inner sep=0,
      outer sep=0.2mm,
      minimum size=0.5mm,
      fill=##1
    }
  }
  
  % >> Progress bar <<  
  \def\progressline@xpad@west{1mm}
  \def\progressline@xpad@east{2mm}
  \def\footer@text@yshift{1mm}
  \pgfdeclarelayer{bg}
  \pgfsetlayers{bg,main}
  \begin{tikzpicture}[overlay,remember picture,shift={(current page.south west)}]
    %
    % @ Footer text @
    \node[footer text] (text)
    at (\footer@text@xshift,\footer@text@yshift)
    {\slidefooter@text};
    \path let \p1=(text.east) in \pgfextra{
      \xdef\progressline@x{\x1}
      \xdef\left@lim{\x1}
    };
    % @ Page counter @
    \ifthenelse{\equal{\appendix@started}{false}}{
      \tikzmath{
        \current@slide@number = \insertframenumber;
        \total@slide@number = \saved@appendix@start@slide;
      }
    }{
      \tikzmath{
        \current@slide@number = int(\insertframenumber-\saved@appendix@start@slide);
        \total@slide@number = int(\inserttotalframenumber-\saved@appendix@start@slide);
      }
    }
    \node[page counter] (counter) at
    ($(current page.south east)+(-\footer@text@xshift,\footer@text@yshift)$)
    {\current@slide@number/\total@slide@number};
    \path let
    \p1=(counter.east),
    \p2=(counter.north) in \pgfextra{
      \xdef\progressline@y{\y1}
      \xdef\right@lim{\x1-5mm}
      \xdef\pagecounter@top{\y2}
    };
    % @ Section checkpoints @
    \pykzexec{section_color_wheel=['beamerGreen','beamerBlue','beamerYellow','beamerRed'];
      n_section_colors=len(section_color_wheel)}
    \tikzmath{
      \width = \right@lim-\left@lim-\progressline@xpad@west-\progressline@xpad@east;
    }
    \coordinate (progress line start) at
    (\progressline@x+\progressline@xpad@west,\progressline@y);    
    \def\section@counter{0}
    \ifthenelse{\equal{\appendix@started}{false}}{
      \xdef\saved@sections{\saved@beamer@sections}
    }{
      \xdef\saved@sections{\saved@beamer@appendices}
    }
    \foreach \section/\section@page in \saved@sections {
      \tikzmath{
        \progress = (\section@page+1)/\total@slide@number*\width;
      }
      \get@section@color{\section@counter}
      \node[section checkpoint=\this@section@color,
      hyperlink node=\section] at
      ($(progress line start)+(\progress*1pt,0)$) {};
      \counter@increment{\section@counter}
    }
    % @ Progress line @
    \tikzmath{
      \progress = \current@slide@number/\total@slide@number*\width;
    }
    \begin{pgfonlayer}{bg}
      \ifthenelse{\equal{\appendix@started}{false}}{
        \draw[progress line] (progress line start) -- ++(\progress*1pt,0) node(tip){};
      }{
        \draw[progress line,appendix line] (progress line start) --
        ++(\progress*1pt,0) node(tip){};
      }
    \end{pgfonlayer}
    \node[progress tip] at (tip) {};
  \end{tikzpicture}

  % >> Section list <<
  \def\list@begin@yshift{0.7em}
  \def\list@dy{1em}
  \begin{tikzpicture}[overlay,remember picture,shift={(current page.south east)}]
    \tikzset{
      sections list element/.style={
        anchor=east,
        scale=0.8
      }
    }
    \def\section@counter{0}
    \ifthenelse{\equal{\appendix@started}{false}}{
      \xdef\saved@sections{\saved@beamer@sections}
      \xdef\@variable{sections}
    }{
      \xdef\saved@sections{\saved@beamer@appendices}
      \xdef\@variable{appendices}
    }
    \foreach \section/\section@page in \saved@sections {
      \tikzmath{
        \current@y = \pagecounter@top+\list@begin@yshift+\section@counter*\list@dy;
      }
      \ifthenelse{\equal{\short@section}{}}{
        \xdef\this@section{0}
      }{
        \pykzmathinline[this@section]{['\short@section' in section
          for section in \@variable].index(True)+1}
      }
      \pgfmathparse{\this@section==\section@counter+1 ? 1:0}
      \ifthenelse{\pgfmathresult>0}{
        \xdef\sectioncolor{beamerRed}
      }{
        \xdef\sectioncolor{beamerGrey}
      }
      \node[sections list element,
      color=\sectioncolor] (section text) at
      (-\footer@text@xshift,\current@y*1pt) {\hyperlink{\section}{\section}};
      \get@section@color{\section@counter}
      \node[section color bubble=\this@section@color,
      anchor=west] at (section text.east) {};
      \counter@increment{\section@counter}
    }
  \end{tikzpicture}
  
}
%%%%%%%%%%%%%%%%%%%%%%%%%%%%%%%%%%%%%%%%%%%%%%%%%%%%%%%%%%%%%%%%%%%%%%%%%%%%%%%%

%%%%%%%%%%%%%%%%%%%%%%%%%%%%%%%%%%%%%%%%%%%%%%%%%%%%%%%%%%%%%%%%%%%%%%%%%%%%%%%%
% ..:: Hyperlinks to slides ::..
\setbeamertemplate{headline}{
  \load@aux
  \ifthenelse{\equal{\short@section}{}}{}{
    \ifthenelse{\equal{\appendix@started}{false}}{
      \xdef\@variable{sections}
    }{
      \xdef\@variable{appendices}
    }
    \pykzexec{start_page = int(\@variable[['\short@section' in section
      for section in \@variable].index(True)].split('/')[1])+1}
    \ifthenelse{\equal{\appendix@started}{false}}{
      \tikzmath{
        \current@slide@number = \insertframenumber;
      }
    }{
      \tikzmath{
        \current@slide@number = int(\insertframenumber-\saved@appendix@start@slide);
      }
    }
    \pykzexec{is_start = start_page==\current@slide@number}
    \pykzmathinline[is@start]{'true' if is_start else 'false'}
    \ifthenelse{\equal{\is@start}{true}}{
      \hypertarget{\short@section}{}
    }{}
  }
}
%%%%%%%%%%%%%%%%%%%%%%%%%%%%%%%%%%%%%%%%%%%%%%%%%%%%%%%%%%%%%%%%%%%%%%%%%%%%%%%% 

%%%%%%%%%%%%%%%%%%%%%%%%%%%%%%%%%%%%%%%%%%%%%%%%%%%%%%%%%%%%%%%%%%%%%%%%%%%%%%%%
% ..:: Slides title ::..
\setbeamercolor{frametitle}{fg=black}
\setbeamerfont{frametitle}{size=\LARGE}
\setbeamertemplate{frametitle}{
  \def\frametitle@yshift{-5mm}
  \tikzset{
    frametitle style/.style={
      scale = 0.8,
      yshift = \frametitle@yshift,
      font = \bfseries
    }
  }
  \begin{tikzpicture}[overlay,remember picture,shift={(current page.north)}]
    \node[frametitle style] at (0,0) {\insertframetitle};
  \end{tikzpicture}
}
%%%%%%%%%%%%%%%%%%%%%%%%%%%%%%%%%%%%%%%%%%%%%%%%%%%%%%%%%%%%%%%%%%%%%%%%%%%%%%%%

%%%%%%%%%%%%%%%%%%%%%%%%%%%%%%%%%%%%%%%%%%%%%%%%%%%%%%%%%%%%%%%%%%%%%%%%%%%%%%%%
% ..:: Page placement ::..
\def\slide@pad@sw{3mm,3mm}
\def\slide@pad@ne{3mm,3mm}
\newcommand{\getframecoord}[2][]{
  % Get absolute coordinate on slide, from fractional coordinate
  % Note that to make sense, the tikzpicture must be shifted to
  % (current page.south west)
  \xdef\frame@count{0}
  \foreach \val in {#2} {
    \pgfmathparse{\frame@count==0 ? 1:0}
    \ifthenelse{\pgfmathresult>0}{
      \xdef\c@x{\val}
    }{
      \xdef\c@y{\val}
    }
    \counter@increment{\frame@count}
  }
  \tikzmath{
    \Lx = \paperwidth/2;
    \Ly = \paperheight/2;
    \@x = \Lx*(1+\c@x);
    \@y = \Ly*(1+\c@y);
  }
  \xdef\framecoord{\@x*1pt,\@y*1pt}
  \ifthenelse{\equal{#1}{}}{}{\expandafter\xdef\csname #1\endcsname{\framecoord}}%
}
\newcommand{\beamergridoverlay}{
  % Overlay a grid to aid in element placement
  \begin{tikzpicture}[overlay,remember picture,shift={(current page.south west)}]
    \def\grid@dx{0.2}
    \def\grid@dy{0.2}
    \tikzmath{
      \x@next = -1+\grid@dx;
      \y@next = -1+\grid@dx;
    }
    \tikzset{
      grid line/.style={
        draw = black!30,
        line width=0.1mm,
        opacity=0.5
      },
      axis tick label/.style={
        text = black!30,
        fill = white,
        inner sep=0.5mm,
        scale = 0.8
      }
    }
    % X lines
    \foreach \x in {-1,\x@next,...,1} {
      \getframecoord[coord@bottom]{\x,-1}
      \pgfmathparse{abs(\x)<1e-3 ? 1:0}
      \ifthenelse{\pgfmathresult>0}{
        \draw[grid line,dashed] (\coord@bottom) -- ++(0,\paperheight);
      }{
        \draw[grid line] (\coord@bottom) -- ++(0,\paperheight);
      }
    }
    % Y lines
    \foreach \y in {-1,\y@next,...,1} {
      \getframecoord[coord@bottom]{-1,\y}
      \pgfmathparse{abs(\y)<1e-3 ? 1:0}
      \ifthenelse{\pgfmathresult>0}{
        \draw[grid line,dashed] (\coord@bottom) -- ++(\paperwidth,0);
      }{
        \draw[grid line] (\coord@bottom) -- ++(\paperwidth,0);
      }
    }
    % X labels
    \foreach \x in {-1,\x@next,...,1} {
      \pgfmathparse{abs(\x)>1e-3 && abs(\x)<1-1e-3 ? 1 : 0}
      \ifthenelse{\pgfmathresult>0}{
        \getframecoord[coord@bottom]{\x,0}
        \pgfmathprintnumberto[precision=2]{\x}{\x@rounded}
        \node[axis tick label] at (\coord@bottom) {\x@rounded};
      }{}
    }
    % Y labels
    \foreach \y in {-1,\x@next,...,1} {
      \pgfmathparse{abs(\y)>1e-3 && abs(\y)<1-1e-3 ? 1 : 0}
      \ifthenelse{\pgfmathresult>0}{
        \getframecoord[coord@bottom]{0,\y}
        \pgfmathprintnumberto[precision=2]{\y}{\y@rounded}
        \node[axis tick label] at (\coord@bottom) {\y@rounded};
      }{}
    }
  \end{tikzpicture}
}
%%%%%%%%%%%%%%%%%%%%%%%%%%%%%%%%%%%%%%%%%%%%%%%%%%%%%%%%%%%%%%%%%%%%%%%%%%%%%%%%

%%%%%%%%%%%%%%%%%%%%%%%%%%%%%%%%%%%%%%%%%%%%%%%%%%%%%%%%%%%%%%%%%%%%%%%%%%%%%%%%
% ..:: Default TikZ environment ::..
\newenvironment{beamertikzpicture}[1][]{%
  \tikzset{%
    apply style/.code={%
      \tikzset{#1}%
    }%
  }%
  \begin{tikzpicture}[overlay,
    remember picture,
    shift={(current page.south west)},
    apply style/.expand once=#1]
}{\end{tikzpicture}}
%%%%%%%%%%%%%%%%%%%%%%%%%%%%%%%%%%%%%%%%%%%%%%%%%%%%%%%%%%%%%%%%%%%%%%%%%%%%%%%%

%%%%%%%%%%%%%%%%%%%%%%%%%%%%%%%%%%%%%%%%%%%%%%%%%%%%%%%%%%%%%%%%%%%%%%%%%%%%%%%%
% ..:: Itemize bullets ::..
\tikzset{
  beamer bullet point/.style={
    circle,
    fill=#1,
    minimum size=0.35em
  }
}
\def\item@y@offset{0.3em}
\setlist[itemize,1]{label={
    \begin{tikzpicture}[inner sep=0,outer sep=0,minimum size=0]
      \node at (0,0) {};
      \node[beamer bullet point=beamerDarkBlue] at (0,\item@y@offset) {};
    \end{tikzpicture}
  }}
\setlist[itemize,2]{label={
    \begin{tikzpicture}[inner sep=0,outer sep=0,minimum size=0]
      \node at (0,0) {};
      \node[beamer bullet point=beamerBlue] at (0,\item@y@offset) {};
    \end{tikzpicture}
  }}
\setlist[itemize,3]{label={
    \begin{tikzpicture}[inner sep=0,outer sep=0,minimum size=0]
      \node at (0,0) {};
      \node[beamer bullet point=beamerGreen] at (0,\item@y@offset) {};
    \end{tikzpicture}
  }}
%%%%%%%%%%%%%%%%%%%%%%%%%%%%%%%%%%%%%%%%%%%%%%%%%%%%%%%%%%%%%%%%%%%%%%%%%%%%%%%% 

\makeatother

%%% Local Variables:
%%% mode: latex
%%% TeX-master: "../main"
%%% End:
