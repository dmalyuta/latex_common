% No paragraph indentation
\setlength\parindent{1em}

% Paragraph spacing
\setlength{\parskip}{0pt}

\def\PtUnit{pt}

\def\innerSep{7cm}
\def\outerSep{1.5cm}
\def\topSep{2cm}
\def\botSep{2cm}
\def\chapterTopSep{2*\topSep}
\def\chapterBotMargin{2cm}

\tikzmath{
  \sidebarWidth = 0.9*\innerSep;
  \textMarginWidth = \innerSep-\sidebarWidth;
  \sidebarPad = 0.5cm;
  \topLabelSep = 2mm;
  \figureWidthLoc = \sidebarWidth-2*\sidebarPad;
}

\def\figureWidth{\figureWidthLoc\PtUnit}
\def\textMarginWidthUnits{\textMarginWidth\PtUnit}

% Expensive figure blocker
\xdef\bookCompileFigures{true}
\newcommand{\makeFigureToggled}[2][]{%
  \ifthenelse{\equal{\bookCompileFigures}{true}}{%
    #2%
  }{%
    \ifthenelse{\equal{#1}{}}{}{%
      \includegraphics[width=#1]{example-image-a}% 
    }{%
      \includegraphics[width=5cm]{example-image-a}%
    }%
  }%
}
\newcommand{\bookToggleFigures}{
  \ifthenelse{\equal{\bookCompileFigures}{true}}{
    \xdef\bookCompileFigures{false}
  }{
    \xdef\bookCompileFigures{true}
  }
}

% Colors
\definecolor{maroon}{rgb}{0.5, 0.0, 0.0}
\newcommand{\highlightcolor}{maroon}
\newcommand{\bookHighlight}[1]{{\itshape\color{\highlightcolor}#1}}
\xdef\sidebarColorNormal{black!10}

% Page types:
%  toc = table of contents
%  text = body text
%  biblio = references list
%  nom = nomenclature
\xdef\sectionType{text}

\usepackage[
% paperwidth=6in,
% paperheight=9in,
letterpaper,
inner=\innerSep,
outer=\outerSep,
top=\topSep,
bottom=\botSep]{geometry}

\pagestyle{fancy}

\fancyhf{} % clear header and footer
\renewcommand{\headrulewidth}{0pt} % no top line

% Figure caption style
\captionsetup[figure]{labelfont={sc},labelformat={default},labelsep=period,name={Figure}}

% Chapter, section, etc. names
\newcommand{\sectiontitle}{}
\newcommand{\setsectiontitle}[1]{\renewcommand{\sectiontitle}{#1}}
\newcommand{\invisiblesection}[1]{%
  \phantomsection%
  \stepcounter{section}%
  \addcontentsline{toc}{section}{\protect\numberline{\thesection}#1}%
}
\newcommand{\mySection}[2][visible]{
  \xdef\theSectionName{#2}
  \label{chapter:\thechapter}
  \ifthenelse{\equal{#1}{invisible}}{
    \invisiblesection{\theSectionName}
  }{
    \section{\theSectionName}
  }
  \label{section:\thesection}
}
\xdef\ChapterQuote{}
\newcommand{\myChapter}[3][]{
  \newpage
  \xdef\ChapterQuote{#1}
  \xdef\sectionType{text}
  \xdef\theChapterName{#2}
  \xdef\theSectionName{#3}
  \chapter{\theChapterName}
  \label{chapter:\thechapter}
  \mySection[invisible]{\theSectionName}
  % Reset counters
  \setcounter{Booktheorem}{0}
  \setcounter{Bookdefinition}{0}
  \setcounter{Bookcorollary}{0}
  \setcounter{Booklemma}{0}
  \setcounter{Bookproposition}{0}
  \setcounter{Bookcondition}{0}
  \setcounter{Bookassumption}{0}
  \setcounter{Bookexample}{0}
  % \expandafter\newsavebox\csname ChapterQuote\thechapter\endcsname
  % \begin{lrbox}{\csname ChapterQuote\thechapter\endcsname}
  %   #1
  % \end{lrbox}
}
\NewEnviron{bookAbstract}[1][]{
  \xdef\ChapterQuote{#1}
  \xdef\sectionType{abstract}
  \newpage
  \thispagestyle{plain}
  \xdef\theChapterName{Abstract}
  \chapter{\theChapterName}
  \label{chapter:\thechapter}
  \BODY
  \newpage
}
\newcommand{\myTOC}{
  \xdef\sectionType{toc}
  \tableofcontents
  \thispagestyle{plain}
  \newpage
}
\newcommand{\myLOF}{
  \xdef\sectionType{lof}
  \listoffigures
  \thispagestyle{plain}
  \newpage
}
\newcommand{\myLOT}{
  \xdef\sectionType{lot}
  \listoftables
  \thispagestyle{plain}
  \newpage
}
\newcommand{\myNomenclature}{
  \xdef\sectionType{nom}
  \printnomenclature
  \thispagestyle{plain}
  \newpage
}
\newcommand{\myBib}[1]{
  \xdef\sectionType{biblio}
  \bibliography{#1.bib}
}
\newcommand{\myBlankPage}{
  \null
  \thispagestyle{empty}%
  \addtocounter{page}{-1}%
  \begin{tikzpicture}[remember picture,overlay,shift={(current page.center)}]
    \node at (0,0) {-- This page is intentionally left blank. --};
  \end{tikzpicture}
  \newpage
}

\newcommand{\calcWidth}[1]{\tikzmath{\width=#1*\figureWidth;}%
  \xdef\customFigureWidth{\width\PtUnit}%
}

% \titlespacing*{\chapter}{0pt}{*0}{0pt}
\newlength{\HugeLetterHeight}
\newlength{\NormalLetterHeight}
\settoheight{\HugeLetterHeight}{\normalfont\Huge A}
\settoheight{\NormalLetterHeight}{A}
\tikzmath{
  \ChapterHeightText=-(\HugeLetterHeight+\parskip+\baselineskip-\NormalLetterHeight);
}
\xdef\ChapterHeightText{\ChapterHeightText\PtUnit}
\titleformat{\chapter}
{\normalfont\huge}
{}
{0cm}
{%
  \xdef\IsChapterPage{true}%
  \begin{tikzpicture}[remember picture,overlay,shift=(current page.north)]%
    \tikzset{
      chapter rule/.style={
        black,
        line width=1mm
      },
      every node/.append style={
        inner sep=0,
        outer sep=0
      },
      text node/.append style={
        anchor=north west,
        font={\bfseries},
        text width=\paperwidth-2*\outerSep
      }
    }
    \tikzmath{
      \ruleWidth = \paperwidth-2*\outerSep;
      \chapterHeight = 0.6cm;
    }
    \tikzmath{
      \xloc=-\paperwidth/2+\outerSep;
      \yloc=-\chapterTopSep;
    }
    \draw[chapter rule] (\xloc*1pt,\yloc*1pt) node(top rule){} to ++(\ruleWidth*1pt,0);
    \ifthenelse{\equal{\sectionType}{text}}{
      \node[text node,align=left] (chapter title)
      at ($(top rule.west)+(0,-\chapterHeight*1pt)$)
      {\thechapter\hspace{1cm} \theChapterName};
      \node[text node,text width=\paperwidth-2*\outerSep,
      anchor=north west,align=right] (first section title)
      at ($(chapter title.south west)+(0,-\chapterHeight*1pt)$)
      {\theSectionName};
      \coordinate (bottom rule start) at (first section title.west);
    }{
      \node[text node,align=left] (chapter title)
      at ($(top rule.west)+(0,-\chapterHeight*1pt)$)
      {\ifthenelse{\equal{\sectionType}{abstract}}{%
          Abstract%
        }{%
          \ifthenelse{\equal{\sectionType}{toc}}{%
            Table of Contents%
          }{%
            \ifthenelse{\equal{\sectionType}{lof}}{%
              List of Figures%
            }{%
              \ifthenelse{\equal{\sectionType}{lot}}{%
                List of Tables%
              }{%
                \ifthenelse{\equal{\sectionType}{biblio}}{%
                  Bibliography%
                }{%
                  Nomenclature%
                }%
              }%
            }%
          }%
        }%
      };
      \coordinate (bottom rule start) at (chapter title.west);
    }
    \ifthenelse{\equal{\ChapterQuote}{}}{}{
      \tikzset{
        chapter quote/.append style={
          align=left,
          font={\large\itshape},
          anchor=north west,
          text width=10cm
        }
      }
      \ifthenelse{\equal{\sectionType}{text}}{
        \node[chapter quote] (chapter quote node) at
        (first section title.north west) {\ChapterQuote};
        \path let
        \p1 = (chapter quote node.south west),
        \p2 = (first section title.south west)
        in \pgfextra{
          \tikzmath{
            \botlinexpos = \x1;
            \botlineypos = min(\y1,\y2);
          }
          \coordinate (bottom rule start) at (\botlinexpos*1pt,\botlineypos*1pt);
        };
      }{
        \node[chapter quote,yshift=-\chapterHeight*1pt] (chapter quote node) at
        (chapter title.south west) {\ChapterQuote};
        \coordinate (bottom rule start) at (chapter quote node.south west);
      }
      \xdef\ChapterQuote{}
    }
    \draw[chapter rule]
    ($(bottom rule start)+(0,-\chapterHeight*1pt)$)
    node(bottom rule){} to ++(\ruleWidth*1pt,0);
    \path let
    \p1=(top rule),
    \p2=(bottom rule)
    in \pgfextra{
      \tikzmath{
        \chapterHeadingHeight=\y1-\y2;
      }
      \xdef\chapterHeadingHeight{\chapterHeadingHeight\PtUnit}
    };
  \end{tikzpicture}%
  \tikzmath{%
    \ExtraSkip = (\chapterTopSep-\topSep)*1pt;%
    \TotalSkip = \ChapterHeightText+\ExtraSkip+\chapterHeadingHeight+%
    \chapterBotMargin;%
  }%
  \xdef\TotalSkip{\TotalSkip\PtUnit}%
  \vspace{\TotalSkip}%
}
\tikzmath{\ChapterHeight=\HugeLetterHeight+\parskip;}
\xdef\ChapterHeight{\ChapterHeight\PtUnit}
\titlespacing{\chapter}{0pt}{-\ChapterHeight}{*0}

\renewcommand{\sectionmark}[1]{\markboth{}{\thesection}}
\newcommand{\mySidebar}[2]{
  % #1: which page? LE, RO
  % #2: which style? plain, fancy
  \ifthenelse{\equal{#1}{LO}}{
    \def\sidebarLocation{north west}
  }{
    \def\sidebarLocation{north east}
  }
  \begin{tikzpicture}[
    remember picture,
    overlay,
    shift={(current page.\sidebarLocation)}
    ]
    \ifthenelse{\equal{#2}{plain}}{
      \def\sidebarColor{white}
    }{
      \def\sidebarColor{\sidebarColorNormal}
    }
    \tikzset{
      sidebar/.style={
        rectangle,
        anchor=\sidebarLocation,
        inner sep=0,
        outer sep=0,
        fill=\sidebarColor,
        minimum width=\sidebarWidth,
        minimum height=\paperheight
      },
      section line/.style={
        black,
        line width=0.3mm
      }
    }
    \node[sidebar] (sidebar region) at (0,0) {};
    % Chapter, section
    \ifthenelse{\equal{#2}{fancy}}{
      \ifthenelse{\equal{#1}{LO}}{
        \def\Pos{1}
        \def\Anchor{north east}
        \def\Align{right}
      }{
        \def\Pos{0}
        \def\Anchor{north west}
        \def\Align{left}
      }
      \draw[section line]
      ($(sidebar region.north west)+(\sidebarPad*1pt,-\topSep)$)
      to
      node[pos=\Pos,anchor=\Anchor,
      text width={\sidebarWidth-2*\sidebarPad},
      align=\Align,
      inner sep=0,
      outer sep=0,
      yshift=-\topLabelSep]
      (section label)
      {\Large%
        \ifthenelse{\equal{\sectionType}{toc}}{%
          {\itshape Contents}%
        }{%
          \ifthenelse{\equal{\sectionType}{biblio}}{%
            {\itshape Bibliography}%
          }{%
            \ifthenelse{\equal{\sectionType}{text}}{%
              \ifthenelse{\equal{#1}{LO}}{%
                \ifthenelse{\equal{\thechapter}{0}}{}{%
                  {\bfseries Chapter~\thechapter}\\%
                  {\itshape \theChapterName}%
                }%
              }{%
                {\bfseries Section~\rightmark}\\%
                {\itshape \theSectionName}%
              }%
            }{}%
          }
        }%
      }
      ($(sidebar region.north east)+(-\sidebarPad*1pt,-\topSep)$);
      \path let
      \p1=(section label.north),
      \p2=(section label.south)
      in \pgfextra{
        \tikzmath{
          \sidebarSectionLabelHeight=\y1-\y2;
        }
        \xdef\sidebarSectionLabelHeight{\sidebarSectionLabelHeight\PtUnit}
      };
    }{\xdef\sidebarSectionLabelHeight{0\PtUnit}}
    % Page number
    \ifthenelse{\equal{#1}{LO}}{
      \coordinate (page left) at
      ($(sidebar region.south west)+(\sidebarWidth/2*1pt,\botSep)$);
      \coordinate (page right) at
      ($(sidebar region.south east)+(-\sidebarPad*1pt,\botSep)$);
      \def\anchorPoint{north east}
    }{
      \coordinate (page left) at
      ($(sidebar region.south east)+(-\sidebarWidth/2*1pt,\botSep)$);
      \coordinate (page right) at
      ($(sidebar region.south west)+(\sidebarPad*1pt,\botSep)$);
      \def\anchorPoint{north west}
    }
    \draw[section line]
    (page left)
    to
    node[pos=1,
    anchor=\anchorPoint,
    inner sep=0,
    outer sep=0,
    yshift=-\topLabelSep,
    align=right
    ]{\thepage}
    (page right);
  \end{tikzpicture}
  % >>>>>>> Draw sidebar figures in the list
  \tikzset{
    figure/.append style={
      inner sep=0,
      outer sep=0,
      anchor=north
    }
  }
  % Define starting Y location
  \ifthenelse{\equal{\IsChapterPage}{true}}{
    \xdef\IsChapterPage{false}
    \tikzmath{
      \SidebarFigureYStart=-(\chapterTopSep+\chapterHeadingHeight+
      \chapterBotMargin);
    }
  }{
    \tikzmath{
      \SidebarFigureYStart=-(\topSep+\sidebarSectionLabelHeight+\SidebarFigureVPadding);
    }
  }
  \xdef\SidebarFigureY{\SidebarFigureYStart\PtUnit}
  % Loop through list of figures that are to be drawn
  \xdef\NewSidebarFigureList{}
  \xdef\Overflow{false}
  \foreach \figure in \SidebarContentList {
    \ifthenelse{\equal{\Overflow}{true}}{
      \AddToList{\NewSidebarFigureList}{\figure}
    }{
      \begin{tikzpicture}[
        remember picture,
        overlay,
        shift={(current page.\sidebarLocation)}
        ]
        \ifthenelse{\equal{#1}{LO}}{
          \tikzmath{\XLoc = \sidebarWidth/2;}
        }{
          \tikzmath{\XLoc = -\sidebarWidth/2;}
        }
        % Draw figure
        \edef\GetBox{\expandafter\usebox\csname \figure\endcsname}
        \node[figure,draw=none,opacity=0] (figure body phantom)
        at (\XLoc*1pt,\SidebarFigureY*1pt) {\phantom{\GetBox}};
        \path let
        \p1=(figure body phantom.north),
        \p2=(figure body phantom.south)
        in \pgfextra{
          \tikzmath{
            \height=(\y1-\y2);
            \SidebarFigureYBottom=\SidebarFigureY-\height;
          }
          \xdef\SidebarFigureYBottom{\SidebarFigureYBottom\PtUnit}
        };
        \pgfmathparse{\SidebarFigureYBottom>\SidebarFigureYEnd ? 1:0}
        \ifthenelse{\pgfmathresult>0}{
          \node[figure] (figure body) at (\XLoc*1pt,\SidebarFigureY*1pt)
          {\GetBox};
        }{
          \AddToList{\NewSidebarFigureList}{\figure}
          \xdef\Overflow{true}
        }
        \path let
        \p1=(figure body.north),
        \p2=(figure body.south)
        in \pgfextra{
          \tikzmath{\height=(\y1-\y2);}
          % Update Y
          \tikzmath{\SidebarFigureY=\SidebarFigureY-\height-\SidebarFigureYSep;}
          \xdef\SidebarFigureY{\SidebarFigureY\PtUnit}
        };
      \end{tikzpicture}
    }
  }
  \xdef\SidebarContentList{\NewSidebarFigureList}
}

\fancyhead[LO]{\mySidebar{LO}{fancy}}
\fancyhead[RE]{\mySidebar{RE}{fancy}}

\fancypagestyle{plain}{
  \fancyhead[LO]{\mySidebar{LO}{plain}}
  \fancyhead[RE]{\mySidebar{RE}{plain}}
}

\let\cleardoublepage=\clearpage
\usepackage{etoolbox}
\makeatletter
\patchcmd{\chapter}{\if@openright\cleardoublepage\else\clearpage\fi}{}{}{}
\makeatother

% Sidebar figure list

\xdef\SidebarContentList{}
\xdef\SidebarFigureCounter{1}
\xdef\SidebarFigureVPadding{1cm}
\xdef\SidebarFigureYSep{1cm}
\tikzmath{
  \SidebarFigureYEnd=-\paperheight+\botSep+\SidebarFigureVPadding;
}
\xdef\SidebarFigureYEnd{\SidebarFigureYEnd\PtUnit}
\newlength{\SidebarFigureHeight}

\NewEnviron{sidebarFigure}{
  % Make the box which is the figure content
  \def\SidebarContentName{SidebarFigure\SidebarFigureCounter}
  \edef\CommandFigureName{\csname \SidebarContentName\endcsname}
  \expandafter\newsavebox\CommandFigureName
  \begin{lrbox}{\CommandFigureName}
    \calcWidth{1}%
    \begin{minipage}{\customFigureWidth}%
      \vspace{-\floatsep}%
      \begin{figure}[H]
        \centering
        \BODY%
      \end{figure}
    \end{minipage}%
  \end{lrbox}
  % \begin{lrbox}{\CommandFigureName}\BODY\end{lrbox}
  \global%
  \expandafter\setbox\CommandFigureName%
  \expandafter\box\CommandFigureName
  % Add to list
  \AddToList{\SidebarContentList}{\SidebarContentName}
  % Update counter
  \tikzmath{\SidebarFigureCounter=int(\SidebarFigureCounter+1);}
  \xdef\SidebarFigureCounter{\SidebarFigureCounter}
}

\newcounter{Comment}
\setcounter{Comment}{0}

\makeatletter
\define@key{sidebarCommentKeys}{clabel}{\def\clabel{#1}}%
\define@key{sidebarCommentKeys}{yshift}{\def\yshift{#1}}%
\makeatother
\newcommand{\sidebarComment}[2][]{%
  \setkeys{sidebarCommentKeys}{clabel=,#1}%
  \refstepcounter{Comment}%
  \ifthenelse{\equal{\clabel}{}}{}{\label{footnote:\clabel}}%
  $^{\textnormal{\ref{footnote:\theComment}}}$%
  % Make the box which is the comment content
  \def\SidebarContentName{SidebarFigure\SidebarFigureCounter}%
  \edef\CommandFigureName{\csname \SidebarContentName\endcsname}%
  \expandafter\newsavebox\CommandFigureName%
  \begin{lrbox}{\CommandFigureName}%
    \calcWidth{1}%
    \begin{minipage}{\customFigureWidth}%
      \protect\label{footnote:\theComment}$^{\theComment}$ #2%
    \end{minipage}%
  \end{lrbox}%
  \global%
  \expandafter\setbox\CommandFigureName%
  \expandafter\box\CommandFigureName%
  % Add to list
  \AddToList{\SidebarContentList}{\SidebarContentName}%
  % Update counter
  \tikzmath{\SidebarFigureCounter=int(\SidebarFigureCounter+1);}%
  \xdef\SidebarFigureCounter{\SidebarFigureCounter}%
}

\newcommand{\AddToList}[2]{%
  % #1: list variable
  % #2: element
  \ifthenelse{\equal{#1}{}}{\xdef#1{#2}}{\xdef#1{#1,#2}}%
}

% Math theorem envs
\def\thmVSpace{3mm}%

\newcommand{\bookProof}[1][]{\vspace{\thmVSpace}\noindent{\itshape Proof%
    \ifthenelse{\equal{#1}{}}{}{~(#1)}.}\hspace{4mm}}

\makeatletter
\define@key{theoremKeys}{type}{\def\type{#1}}%
\define@key{theoremKeys}{tlabel}{\def\tlabel{#1}}%
\makeatother

\newcounter{Booktheorem}
\newcounter{Bookdefinition}
\newcounter{Bookcorollary}
\newcounter{Booklemma}
\newcounter{Bookproposition}
\newcounter{Bookcondition}
\newcounter{Bookassumption}
\newcounter{Bookexample}
\newcounter{BookThmCounter}
\setcounter{Booktheorem}{0}
\setcounter{Bookdefinition}{0}
\setcounter{Bookcorollary}{0}
\setcounter{Booklemma}{0}
\setcounter{Bookproposition}{0}
\setcounter{Bookcondition}{0}
\setcounter{Bookassumption}{0}
\setcounter{Bookexample}{0}
\setcounter{BookThmCounter}{1}

\newcounter{qedCounter}
\setcounter{qedCounter}{1}
\renewcommand{\qedsymbol}{}
\newcommand{\qedmark}{%
  \pgfmark{qedmark\theqedCounter}%
  \stepcounter{qedCounter}%
}

\NewEnviron{bookTheorem}[1][]{%
  % Parameters%
  \setkeys{theoremKeys}{type=theorem,#1}%
  \setkeys{theoremKeys}{tlabel=,#1}%
  % HEADING
  \ifthenelse{\equal{\type}{}}{}{%
    \edef\CounterName{Book\type}%
    \refstepcounter{\CounterName}%
    \edef\name{\capitalisewords{\type}~\arabic{\CounterName}}%
    \label{\type:\tlabel}%
    % Title
    \vspace{\thmVSpace}%
    \edef\markNameStart{defStart\theBookThmCounter}%
    \edef\markNameEnd{defEnd\theBookThmCounter}%
    {\noindent\pgfmark{\markNameStart}\phantom{\bfseries\name.}}\pgfmark{\markNameEnd}
    \begin{tikzpicture}[remember picture,overlay]%
      \tikzmath{
        \pad = 1mm;
        \height = 0.7em+2*\pad;
        \Angle = 60;
        \tipOffset = \height*tan(90-\Angle);
        \tipOffsetValueLocal = 2.5*\tipOffset;
      }
      \xdef\tipOffsetValue{\tipOffsetValueLocal\PtUnit}
      \tikzset{
        theorem background/.append style={
          fill=black!10,
          draw=black!30,
          line width=0.4mm
        }
      }
      \coordinate (south west) at (pic cs:\markNameStart);
      \coordinate (south east) at (pic cs:\markNameEnd);
      \draw[theorem background]
      ($(south west)+(0,-\pad*1pt)$) to
      ($(south east)+(2*\pad*1pt,-\pad*1pt)$) to
      ++({(\tipOffset+\pad)*1pt},\height*1pt) to
      ($(south west)+(0,-\pad*1pt)+(0,\height*1pt)$) to
      cycle;
      \node[anchor=north west,inner sep=0,outer sep=0,font={\bfseries}] at ($(south west)+(\pad*1pt,\height*1pt-2*\pad*1pt)$) {\name.};
    \end{tikzpicture}%
    \hspace{\tipOffsetValue}%
  }%
  % BODY
  \BODY\ifthenelse{\equal{\theqedCounter}{\theBookThmCounter}}{\qedmark}{}%
  \stepcounter{BookThmCounter}%
  \addtocounter{qedCounter}{-1}%
  % QED
  \begin{tikzpicture}[remember picture,overlay]%
    \def\qedSz{0.7em}
    \tikzset{
      QED/.style={
        anchor=south west,
        rectangle,
        inner sep=0,
        outer sep=0,
        minimum size=\qedSz,
        fill=white,
        draw=black
      }
    }
    \path let
    \p1=(pic cs:qedmark\theqedCounter),
    \p2=(current page.east)
    in \pgfextra{
      \ifthispageodd{
        \tikzmath{\xloc=\x2-\outerSep+\qedSz;}
      }{
        \tikzmath{\xloc=\x2-\innerSep+\qedSz;}
      }
      \node[QED] at (\xloc*1pt,\y1) {};
    };
  \end{tikzpicture}%
  \addtocounter{qedCounter}{1}%
  \vspace{\thmVSpace}%
}

% List style
\def\EnumMargin{2cm}
\NewEnviron{bookList}{
  \begin{enumerate}[
    leftmargin=\textMarginWidthUnits,
    rightmargin=\textMarginWidthUnits,
    align=left,
    labelwidth=0.7em,
    labelsep=0pt,
    label=\protect\fancyarrow
    ]
    \BODY
  \end{enumerate}
}

% Bibliography style
\setlength{\bibsep}{1pt}
\def\bibfont{\footnotesize}
\gappto{\UrlBreaks}{\UrlOrds}

%%% Local Variables:
%%% mode: latex
%%% TeX-master: t
%%% End:
