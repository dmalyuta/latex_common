% ..:: Generic math ::..

% Parentheses
\ExplSyntaxOn
\makeatletter
\NewDocumentCommand{\parengen}{m m m m}
{
  % \group_begin: Makes the variables inside *local* to the group
  % Source: https://www.alanshawn.com/latex3-tutorial/
  \group_begin:
  % Parameters
  \tl_set:Nn \l_left_paren_tl {#1}
  \tl_set:Nn \l_right_paren_tl {#2}
  \tl_set:Nn \l_paren_sz_tl {#3}
  % Logic
  \ifthenelse{\equal{\l_paren_sz_tl}{}}{
    \l_left_paren_tl #4 \l_right_paren_tl
  }{
    \ifthenelse{\equal{\l_paren_sz_tl}{auto}}{
      \left\l_left_paren_tl #4 \right\l_right_paren_tl
    }{
      \csname \l_paren_sz_tl l\endcsname\l_left_paren_tl
      #4
      \csname \l_paren_sz_tl r\endcsname\l_right_paren_tl
    }
  }
  \group_end:
}

\newcommand{\pare}[2][]{\parengen{(}{)}{#1}{#2}}
\newcommand{\brak}[2][]{\parengen{[}{]}{#1}{#2}}
\newcommand{\brac}[2][]{\parengen{\{}{\}}{#1}{#2}}
\newcommand{\angl}[2][]{\parengen{\langle}{\rangle}{#1}{#2}}
\newcommand{\abso}[2][]{\parengen{|}{|}{#1}{#2}}

\newcommand{\@@normgen}[2][]{\parengen{\|}{\|}{#1}{#2}}

\makeatother
\ExplSyntaxOff

% Ifthenelse based on a mathematical condition
\newcommand{\ifelsemath}[3]{%
  \pgfmathparse{#1 ? 1:0}%
  \ifthenelse{\pgfmathresult>0}{#2}{#3}}

\newcommand{\where}{:}
\newcommand{\given}{\mid}
\newcommand{\CC}[1]{\mathcal C^{#1}}
\newcommand{\inv}[1][]{\ifthenelse{\equal{#1}{}}{^{-1}}{^{-#1}}}
\newcommand{\invT}{^{-\transp}}
\newcommand{\invsq}{^{-\frac{1}{2}}}
\newcommand{\polar}{^{\circ}}
\newcommand{\ortho}{^{\perp}}
\newcommand{\powsq}{^{\frac{1}{2}}}
\newcommand{\ceil}[1]{\left\lceil#1\right\rceil}
\newcommand{\floor}[1]{\left\lfloor#1\right\rfloor}
\newcommand{\sign}{\mathrm{sgn}}
\ExplSyntaxOn
\makeatletter
% Sources inspiring the below:
%   - Main: https://tex.stackexchange.com/a/326918/90535
%   - Optional arg count (nargin): https://tex.stackexchange.com/a/395784/90535
\NewDocumentCommand{\norm}{o m}
{
  \group_begin:
  % split the input at commas
  \IfNoValueTF{#1}{
    \def\@nargin{0}
  }{
    \seq_set_split:Nnn \l_var_seq { , } { #1 }
    \edef\@nargin{\seq_count:N \l_var_seq}
  }
  % set the item to variables
  \seq_pop_left:NN \l_var_seq \l_norm_type_tl
  \seq_pop_left:NN \l_var_seq \l_delimiter_size_tl
  % Defaults
  \pgfmathparse{\@nargin<=1 ? 1:0}
  \ifthenelse{\pgfmathresult>0}{
    \tl_set:Nn \l_delimiter_size_tl {}
  }{}
  \pgfmathparse{\@nargin==0 ? 1:0}
  \ifthenelse{\pgfmathresult>0}{
    \tl_set:Nn \l_norm_type_tl {2}
  }{}
  % Print the text
  % print; _ is not available here
  \ifthenelse{\equal{\l_delimiter_size_tl}{auto}}{%
    \@@normgen[auto]{#2}}{%
    \ifthenelse{\equal{\l_delimiter_size_tl}{}}{
      \@@normgen[]{#2}}{\@@normgen[\l_delimiter_size_tl]{#2}}}%
  \c_math_subscript_token{\l_norm_type_tl}
  \group_end:
}
\seq_new:N \l_var_seq
\tl_new:N \l_norm_type_tl
\tl_new:N \l_delimiter_size_tl
\makeatother
\ExplSyntaxOff

\newcommand{\dualnorm}[2][]{\norm[#1]{#2}^*}
\newcommand{\vertiii}[1]{{\left\vert\kern-0.25ex\left\vert\kern-0.25ex\left\vert #1
        \right\vert\kern-0.25ex\right\vert\kern-0.25ex\right\vert}}
\newcommand{\inorm}[2][]{\vertiii{#2}_{#1}}
\newcommand{\seq}[1]{\{#1\}}
\newcommand{\sgn}{\mbox{\textrm{sgn}}}
\newlength{\dhatheight}
\newcommand{\hhat}[1]{%
  \settoheight{\dhatheight}{\ensuremath{\hat{#1}}}%
  \addtolength{\dhatheight}{-0.35ex}%
  \hat{\vphantom{\rule{1pt}{\dhatheight}}%
    \smash{\hat{#1}}}}

\makeatletter
\define@key{projKeys}{set}{\def\set{#1}}
\define@key{proxKeys}{t}{\def\t{#1}}
\define@key{proxKeys}{f}{\def\f{#1}}
\makeatother
\DeclareMathOperator{\projecti}{proj}
\DeclareMathOperator{\proj}{proj}
\DeclareMathOperator{\proximal}{prox}
\newcommand{\Proj}[2][]{%
  \setkeys{projKeys}{set=,#1}%
  \mathchoice{\underset{\set}{\projecti}}%
  {\projecti_{\set}}{\projecti_{\set}}{\projecti_{\set}}%
  \left(#2\right)%
}
\newcommand{\Prox}[2][]{%
  \setkeys{proxKeys}{t=t,#1}%
  \setkeys{proxKeys}{f=f,#1}%
  \mathchoice{\underset{\t\f}{\proximal}}%
  {\proximal_{\t\f}}{\proximal_{\t\f}}{\proximal_{\t\f}}%
  \ifthenelse{\equal{#2}{}}{}{\left(#2\right)}%
}
\DeclareMathOperator{\distance}{dist}
\newcommand{\dist}[2][]{%
  \distance_{#1}\ifthenelse{\equal{#2}{}}{}{(#2)}
}
\newcommand{\gauge}[2][]{%
  \gamma_{#1}\ifthenelse{\equal{#2}{}}{}{(#2)}
}

\newcommand{\Pre}{\mathop\text{Pre}}
\newcommand{\PP}{$\mathcal{P}$\xspace}
\newcommand{\NP}{$\mathcal{NP}$\xspace}
\newcommand{\NPhard}{$\mathcal{NP}$-hard\xspace}
\newcommand{\NPcomplete}{$\mathcal{NP}$-complete\xspace}
\newcommand{\transp}{{\scriptscriptstyle\mathsf{T}}}
\newcommand{\T}{^\transp}
\newcommand{\innerproduct}[3][]{\langle#2,#3\rangle_{#1}}
\newcommand{\inner}[2][]{\langle#2\rangle_{#1}}
\newcommand{\dd}[1][]{#1\mathrm{d}}
\newcommand{\dt}[1][]{\dd[#1]t}
\newcommand{\sdd}{\dd[\,]}
\newcommand{\sdt}{\dt[\,]}
\newcommand{\fromabove}{\downarrow}
\newcommand{\frombelow}{\uparrow}
\newcommand{\fun}[2][1]{%
  #2(%
  \foreach \index in {1, ..., #1} {%
    \ifthenelse{\equal{\index}{#1}}{%
      \cdot%
    }{%
      \cdot,%
    }%
  })}
\newcommand{\define}[1]{\textit{#1}}%
\newcommand{\definedas}[1][tri]{%
  \ifthenelse{\equal{#1}{tri}}{\triangleq}{\coloneqq}}
\newcommand{\inlineint}{\textstyle\int}
\newcommand{\inlinesum}{\textstyle\sum}
\newcommand{\evalat}[2]{\left.#1\right|_{#2}}
\newcommand{\epsmach}{\epsilon_{mach}}
\renewcommand{\implies}{\Rightarrow}
\renewcommand{\iff}{\Leftrightarrow}
\newcommand{\mleq}{\preceq}
\newcommand{\mgeq}{\succeq}
\newcommand{\mle}{\prec}
\newcommand{\mgr}{\succ}
\DeclareMathOperator*{\exptx}{exp}
\renewcommand{\exp}[2][exponent]{\ifthenelse{\equal{#1}{exponent}}{e^{#2}}{\exptx\left(#2\right)}}
\DeclareMathOperator*{\expsf}{\mathsf{exp}}
\newcommand{\expm}[2][exponent]{\ifthenelse{\equal{#1}{exponent}}{\mathsf{e}^{#2}}{\expsf\left(#2\right)}}
\newcommand{\ttilde}[1]{\tilde{\raisebox{0pt}[0.85\height]{%
      $\tilde{#1}$}}}
\newcommand{\tttilde}[1]{\tilde{\raisebox{0pt}[0.88\height]{%
      $\tilde{\raisebox{0pt}[0.85\height]{%
          $\tilde{#1}$}}$}}}
\newcommand{\overlline}[1]{\overline{\overline{#1}}}
\newcommand{\bbar}[1]{\bar{\bar{#1}}}
\newcommand{\prefix}[2][]{{}^{#1}_{#2}}

\makeatletter
\DeclareFontFamily{U}{tipa}{}
\DeclareFontShape{U}{tipa}{m}{n}{<->tipa10}{}
\newcommand{\arc@char}{{\usefont{U}{tipa}{m}{n}\symbol{62}}}%
\renewcommand{\arc}[1]{\mathpalette\arc@arc{#1}}
\newcommand{\arc@arc}[2]{%
  \sbox0{$\m@th#1#2$}%
  \vbox{
    \hbox{\resizebox{\wd0}{\height}{\arc@char}}
    \nointerlineskip
    \box0
  }%
}
\makeatother

\makeatletter
\newenvironment{taggedsubequations}[1]
{%
  % \end{subequations} will advance `equation`
  \addtocounter{equation}{-1}%
  \begin{subequations}%
    % set the current label
    \def\@currentlabel{#1}%
    % redefine \theequation
    \renewcommand{\theequation}{#1.\alph{equation}}%
  }
  {\end{subequations}}
\makeatother %not \makeatletter

\newcommand{\dash}{\Hyphdash}

% ..:: Statistics ::..

\DeclareMathOperator{\expectation}{\mathbf{E}}
\newcommand{\EE}{\expectation}
\DeclareMathOperator{\variance}{\mathbf{Var}}
\newcommand{\Var}{\variance}
\DeclareMathOperator{\prob}{\mathrm{prob}}

% ..:: Sets ::..

\newcommand{\powerSet}[1]{2^{#1}}
\newcommand{\union}{\cup}
\newcommand{\intersection}{\cap}
\newcommand{\Union}{\bigcup}
\newcommand{\Intersection}{\bigcap}
\newcommand{\set}[1]{\mathcal #1}
\newcommand{\setcomplement}[1]{#1^{\mathrm{c}}}
\newcommand{\eucledian}[1][]{\ifthenelse{\equal{#1}{}}{\mathbb E}{\mathbb #1}}
\DeclareMathOperator{\@convhull}{conv}
\newcommand{\reals}[1][]{%
  \ifthenelse{\equal{#1}{extended}}{\overline}{}{\mathbb R}%
}
\newcommand{\extendedreals}{\overline\reals} % DEPRECATED
\newcommand{\real}{\mathbb R}
\newcommand{\pos}[1][\reals]{#1_{++}}
\newcommand{\nonneg}[1][\reals]{#1_{+}}
\newcommand{\normball}[1][]{%
  % {x : ||x||_#1 <= 1}
  % #1: norm type
  \mathbb B_{#1}}
\newcommand{\integers}{\mathbb Z}
\newcommand{\naturals}{\mathbb N}
\newcommand{\binary}{\mathbb I}
\newcommand{\quaternion}{\mathbb Q}
\newcommand{\symm}{\mathbb S}  % symmetric matrices
\newcommand{\psd}{\symm_{+}}
\newcommand{\pd}{\symm_{++}}
\DeclareMathOperator{\dom}{dom}
\DeclareMathOperator{\epi}{epi}
% \newcommand{\hypo}{\mathrm{hypo}}
% \newcommand{\interior}[1]{\mathrm{int}(#1)}
\DeclareMathOperator{\interior}{int}
\DeclareMathOperator{\relint}{relint}
\DeclareMathOperator{\aff}{aff}
\DeclareMathOperator{\bd}{bd}
\DeclareMathOperator{\closure}{cl}
\newcommand{\compl}[1]{#1^\mathrm{c}}
\DeclareMathOperator{\boundary}{\partial}
% \newcommand{\relint}{\mathrm{relint}}
\newcommand{\coni}[1]{\mathrm{coni}(#1)}
\newcommand{\bdry}{\partial}
\newcommand{\image}[1]{\mathrm{im}(#1)}
\newcommand{\setHtwo}{\set{H}_2}
\newcommand{\setLinfty}{\set{L}_\infty}
\newcommand{\Ellipsoid}[1][]{\mathcal E_{#1}}
\newcommand{\sequence}[3][k]{\{#2_{#1}\}_{#1=1}^{#3}}
\newcommand{\minkowplus}{\oplus}
\newcommand{\powerset}[1]{\mathscr{P}(#1)}
\newcommand{\zeroset}{\{0\}}
\DeclareMathOperator{\vol}{vol}
\newcommand{\volume}[1]{\vol\left(#1\right)}
\DeclareMathOperator{\vtcs}{\set V}
\newcommand{\vertices}[1]{\vtcs\left(#1\right)}
\DeclareMathOperator{\myfixedpointset}{fixed}
\newcommand{\fixedpoint}[1]{\myfixedpointset\left(#1\right)}
\DeclareMathOperator{\sublevelsetsymbol}{lev}
\newcommand{\levelset}[2][]{\sublevelsetsymbol_{#1}#2}

% ..:: Algorithms ::..

\newcommand{\algvar}[1]{\text{\IfSubStr{#1}{_}{%
    \StrSubstitute{#1}{_}{\textunderscore}}{#1}}}
\newcommand{\true}{\algvar{true}\xspace}
\newcommand{\false}{\algvar{false}\xspace}
\newcommand{\bigO}[1]{O\left(#1\right)}
\newcommand{\smallO}[1]{o\left(#1\right)}
\newcommand{\bigOmega}[1]{\Omega\left(#1\right)}
\newcommand{\bigTheta}[1]{\Theta\left(#1\right)}
\newcommand{\kd}{$k$-d\xspace}

% ..:: Computer science ::..

\makeatletter
\definecolor{listinggray}{gray}{0.9}
\definecolor{lbcolor}{rgb}{0.9,0.9,0.9}
\colorlet{Darkgreen}{green!60!black}
\lstdefinelanguage{Julia}%
{morekeywords={abstract,break,case,catch,const,continue,do,else,elseif,%
    end,export,false,for,function,immutable,import,importall,if,in,%
    macro,module,otherwise,quote,return,switch,true,try,type,typealias,%
    using,while},%
  sensitive=true,%
  alsoother={\$},%\$
  morecomment=[l]\#,%
  morecomment=[n]{\#=}{=\#},%
  morestring=[s]{"}{"},%
  morestring=[m]{'}{'},%
}[keywords,comments,strings]
%
\lstset{
  basicstyle=\scriptsize,
  backgroundcolor=\color{lbcolor},
  tabsize=4,
  aboveskip={1.5\baselineskip},
  breaklines=true,
  upquote=true,
  prebreak = \raisebox{0ex}[0ex][0ex]{\ensuremath{\hookleftarrow}},
  frame=single,
  numbers=left,
  showtabs=false,
  showspaces=false,
  showstringspaces=false,
  identifierstyle=\ttfamily,
  keywordstyle=\color[rgb]{0,0,1},
  commentstyle=\color{Darkgreen},
  stringstyle=\color{red},
  numberstyle=\color[rgb]{0.205, 0.142, 0.73},
}
%
\lstdefinestyle{listing@C++}{
  language=[GNU]C++,
}
%
\lstdefinestyle{listing@Julia}{%
  language = Julia,
}
%
\lstdefinestyle{listing@Python}{%
  language = Python,
}
%
\newcommand\langname@bash{}
\def\langname@bash{bash}
\newcommand\prompt@bash{\texttt{\$}\ } %\$
\newcommand\addedToEveryPar@bash{}
\lst@AddToHook{EveryPar}{\addedToEveryPar@bash}
\lst@AddToHook{PreInit}{%
  \ifx\lst@language\langname@bash%
  \let\addedToEveryPar@bash\prompt@bash%
  \fi
}
\lstdefinestyle{listingterminal}{%
  language=bash,
  backgroundcolor=\color{black!90!blue},
  basicstyle=\color{white},
  commentstyle=\color{yellow!50},
  keywordstyle=\bfseries\color{orange},
  breaklines,
  escapechar=@
}
%
\newcommand{\cppcode}[2][]{\lstinputlisting[style=listing@C++,#1]{#2}}
\newcommand{\pythoncode}[2][]{\lstinputlisting[style=listing@Python,#1]{#2}}
\newcommand{\juliacode}[2][]{\lstinputlisting[style=listing@Julia,#1]{#2}}
\lstnewenvironment{terminalcode}[1][]{\lstset{style=listingterminal,#1}}{}
\makeatother

% ..:: Analysis ::..

\newcommand{\Lone}[1]{\|#1\|_1}
\newcommand{\Ltwo}[1]{\|#1\|_2}
\newcommand{\FLinf}[2]{\|#1\|_{#2}}
\newcommand{\converge}[1][]{\overset{#1}{\to}}

% ..:: Numerical analysis ::..

\renewcommand{\O}[1]{\mathcal O(#1)}

% ..:: Calculus ::..

\let\oldprime\prime
\renewcommand{\prime}{^\oldprime}
\newcommand{\pprime}{''}
\newcommand{\ppprime}{'''}
\newcommand{\grad}{\nabla}
\newcommand{\hess}{\nabla^2}
\newcommand{\diff}[2]{\nabla_{#1}#2}
\newcommand{\hot}{\textnormal{h.o.t.}}
\newcommand{\ddx}{\frac{\dd}{\dd x}}

% ..:: Convex analysis ::..

\newcommand{\subdiff}[1][none]{%
  \ifthenelse{\equal{#1}{none}}{%
    \partial%
  }{%
    \partial_{#1}%
  }%
}
\newcommand{\normalcone}[2]{\set{N}_{#1}(#2)}
\DeclareMathOperator{\conehull}{cone}
\makeatletter
\DeclareMathOperator{\@convhull}{conv}
\newcommand{\convhull}[1][]{%
  \ifthenelse{\equal{#1}{closed}}{\overline}{}\@convhull%
}
\DeclareMathOperator{\@closed@convex@envelope}{\overline{co}}
\newcommand{\cce}{\@closed@convex@envelope}
\makeatother
\DeclareMathOperator{\extreme}{ext}
\newcommand{\indicator}[1]{\delta_{#1}}
\newcommand{\support}[2][delta]{\ifthenelse{\equal{#1}{delta}}{%
    \delta^*_{#2}%
  }{\sigma_{#2}}}
\newcommand{\conj}{^{*}}
\newcommand{\conjj}{^{**}}
\newcommand{\shuffle}{^{\uparrow}}

% ..:: Functions ::..

\newcommand{\heaviside}{\sigma}

% ..:: Linear Algebra ::..

\DeclareMathOperator{\nul}{null}
\DeclareMathOperator{\rank}{rank}
\DeclareMathOperator{\range}{range}
\DeclareMathOperator{\gph}{gph}
\DeclareMathOperator{\img}{img}
\DeclareMathOperator{\linspan}{span}
\DeclareMathOperator{\diag}{diag}
\DeclareMathOperator{\spec}{spec}
\DeclareMathOperator{\tr}{\mathbf{tr}}
\DeclareMathOperator{\vectorize}{vec}
\newcommand{\Spec}[1]{\spec\left(#1\right)}
\newcommand{\rpinv}{^\dagger}
\newcommand{\lpinv}{^\ddagger}
\newcommand{\blkdiag}[1]{\mathrm{blkdiag}(#1)}
\DeclareMathOperator{\trace}{\mathbf{tr}}
\newcommand{\svmax}[1]{\bar\sigma(#1)}
\newcommand{\Matrix}[2][]{%
  \ifthenelse{\equal{#1}{}}{}{\setlength\arraycolsep{#1}}%
  \begin{bmatrix}#2\end{bmatrix}}
\renewcommand{\Array}[2][]{%
  \ifthenelse{\equal{#1}{}}{}{\setlength\arraycolsep{#1}}%
  \begin{matrix}#2\end{matrix}}
\newcommand{\Mnl}{\protect\\} % protected newline
\newcommand{\linmap}{\mathcal}
\makeatletter
\newcommand{\toset}{%
  \def\arr@offset{0.15em}
  \def\arr@len{0.7em}
  \def\arr@height{0.3em}
  \tikz[minimum height=0ex,outer sep=0,inner sep=0]
  \path[-{Latex[length=0.8mm]}]
  node (a) at (0,0) {}
  node (b) at (0,\arr@height) {}
  (a) edge ++(\arr@len,0)
  (b) edge ++(\arr@len,0)
  (a) edge[draw=none] ++(0,-\arr@offset);%
}
\makeatother
\newcommand{\zeros}{\bm{0}}
\newcommand{\ones}{\bm{1}}

% ..:: Geometry ::..

\newcommand{\fr}[1]{\mathcal{F}_{\mathcal{#1}}}

% ..:: Measure theory ::..

\newcommand{\alev}[1]{\text{a.e. }#1}
\newcommand{\gen}[1]{\text{gen. }#1}
\newcommand{\ev}[3][c]{%
  \ifthenelse{\equal{#1}{c}}{%
    \ifthenelse{\equal{#2}{}}{\forall[0,#3]}{\forall[#2,#3]}%
  }{%
    \ifthenelse{\equal{#1}{o}}{%
      \ifthenelse{\equal{#2}{}}{\forall(0,#3)}{\forall(#2,#3)}%
    }{%
      \ifthenelse{\equal{#1}{oc}}{%
        \ifthenelse{\equal{#2}{}}{\forall(0,#3]}{\forall(#2,#3]}%
      }{%
        \ifthenelse{\equal{#2}{}}{\forall[0,#3)}{\forall[#2,#3)}%
      }%
    }%
  }%
}

% ..:: Optimization ::..

\DeclareMathOperator*{\argmax}{argmax}
\DeclareMathOperator*{\argmin}{argmin}
\DeclareMathOperator*{\minimize}{\mathrm{minimize}}
\DeclareMathOperator*{\maximize}{\mathrm{maximize}}
\DeclareMathOperator*{\find}{\mathrm{find}}
\DeclareMathOperator{\val}{val}

% Get more optimization commands via \makeatletter

\def\emptyCaption{{\color{white}?}}

\def\optiPadding{1mm}
\def\captionPadding{1mm}

\newcommand{\clonelabel}[2]{\@bsphack
  \expandafter\ifx\csname r@#2\endcsname\relax
  \else\protected@write\@auxout{}{\string\newlabel{#1}%
    {\csname r@#2\endcsname}}%
  \fi
  \expandafter\ifx\csname r@#2@cref\endcsname\relax
  \else\protected@write\@auxout{}{\string\newlabel{#1@cref}%
    {\csname r@#2@cref\endcsname}}%
  \fi
  \@esphack}

\newcounter{ProblemCounter}
\newcounter{EqStartCounter}
\newcounter{ll}
\newcounter{jj}
\newcounter{kk}

\setcounter{ProblemCounter}{0}
\setcounter{EqStartCounter}{0}

\newcommand{\setBound}[2][true]{%
  % #1: apply \tikzmark if true
  % #2: tikzmark label
  \ifthenelse{\equal{#1}{true}}{
    \tikzmark{#2\theProblemCounter}
  }{}
}
\newcommand{\getBound}[1]{pic cs:#1\theProblemCounter}

\newlength{\lineHeight}

\newcommand{\theBackground}[5]{%
  % #1: \border
  % #2: type {cost,constraint}
  % #3: cost OR constraint list
  % #4: constraint number (only relevant if type=constraint)
  % #5: \@caption
  \begin{tikzpicture}[remember picture,overlay]%
    % Styles
    \ifthenelse{\equal{#1}{true}}
    {\def\opacityValue{1}}
    {\def\opacityValue{0}}
    \tikzset{
      background/.append style={
        rectangle,
        anchor=south west,
        fill=yellow!20,
        inner sep=0,
        outer sep=0,
        minimum width=\problemWidth+2*\optiPadding,
        opacity=\opacityValue
      },
      outline box/.append style={
        anchor=south west,
        inner sep=0,
        outer sep=0
      },
      border/.append style={
        solid,
        draw=orange!80,
        line width=0.2mm,
        opacity=\opacityValue
      },
      border cap/.append style={
        background,
        border,
        anchor=west,
        minimum height=2*\optiPadding,
        rounded corners=\optiPadding,
        opacity=\opacityValue
      }
    }
    % Cost background
    \ifthenelse{\equal{#2}{cost}}{
      \phantomoutline{#3}
      \edef\costBelowShift{\belowShift}
      \path let
      \p1=(\getBound{cost.south west}),
      \p2=(\getBound{variables.south west})
      in \pgfextra{
        \node[
        background,
        minimum height=\totalHeight
        ] (cost) at (\x1-\optiPadding,\y1-\belowShift) {};
        \draw[border] (cost.south west) to (cost.north west);
        \draw[border] (cost.south east) to (cost.north east);
      };
      % Draw the top cap
      \begin{scope}
        \clip
        ($(cost.north west)+(-\optiPadding,0)$)
        rectangle
        ($(cost.north east)+(2*\optiPadding,2*\optiPadding)$);
        \node[border cap] at (cost.north west) {};
      \end{scope}
      \xdef\lastLine{cost}
      % Caption
      \ifthenelse{\equal{\writeCaption}{true}}{
        \node[
        anchor=south east,
        align=right,
        inner sep=0,
        outer sep=0,
        font={\bfseries\footnotesize\baselineskip=1.3em}
        ] (problemLabel) at ($(cost.south west)+
        (-\captionPadding,\costBelowShift*1pt)$)
        {\ifthenelse{\equal{\@tag}{}}{Problem \theequation.}{Problem \@tag.}};
        \node[
        anchor=north east,
        align=right,
        inner sep=0,
        outer sep=0,
        font={\itshape\footnotesize\baselineskip=1.3em}
        ] (description) at ($(problemLabel.south east)+(0,-2*\captionPadding)$)
        {#5};
      }{}
    }{}
    % Constraint background
    \ifthenelse{\equal{#2}{constraint}}{
      \coordinate (previous) at (\lastLine.south west);
      \phantomoutline{#3[#4]}
      \path let
      \p1 = (\getBound{constraint#4.south west}),
      \p2 = (previous)
      in \pgfextra{
        \tikzset{
          marker/.append style={
            inner sep=0,
            outer sep=0,
            minimum size=1mm,
            fill=black,
            circle
          }
        }
        \ifthenelse{\equal{#4}{\listlen#3[]}}{\tikzmath{\padExtraBelow=0;}}%
        {\tikzmath{\padExtraBelow=\optiPadding;}}
        \pgfmathparse{\y1<\y2 ? 1 : 0}
        \ifthenelse{\pgfmathresult>0}{
          \tikzmath{\lineHeight=\y2-(\y1-\belowShift)+\padExtraBelow;}
        }{
          \tikzmath{\lineHeight=\totalHeight+2*\padExtraBelow;}
        }
        \tikzmath{\constraintBgShiftX=\costResultWidth+\optiPadding;}
        \node[
        background,
        minimum height=\lineHeight
        ] (constraint#4) at (\x1-\constraintBgShiftX,\y1-\belowShift-\padExtraBelow) {};
        \draw[border] (constraint#4.south west) to (constraint#4.north west);
        \draw[border] (constraint#4.south east) to (constraint#4.north east);
      };
      \xdef\lastLine{constraint#4}
      % Draw the bottom cap
      \ifthenelse{\equal{#4}{\listlen#3[]}}{
        \begin{scope}
          \clip
          ($(\lastLine.south west)+(-\optiPadding,0)$)
          rectangle
          ($(\lastLine.south east)+(2*\optiPadding,-2*\optiPadding)$);
          \node[border cap] at (\lastLine.south west) {};
        \end{scope}
      }{}
    }{}
  \end{tikzpicture}%
}

\newcommand{\printCost}[6][true]{
  % #1: print with \setBound and alignment if true
  % #2: \result
  % #3: \constrained
  % #4: \task
  % #5: \variables
  % #6: \objective
  \setBound[#1]{cost.south west}%
  \ifthenelse{\equal{#2}{}}{}{#2=}%
  \setBound[#1]{cost.south west2}%
  \ifthenelse{\equal{#3}{true}}{\ifthenelse{\equal{#1}{true}}{&}{}}{}%
  #4_{\setBound[#1]{variables.south west}#5\setBound[#1]{bottom}}#6%
  \setBound[#1]{cost.south east}%
  \ifthenelse{\equal{#3}{true}}{%
    ~\textnormal{s.t.}\setBound[#1]{cost.south east}
  }{}%
  % determine width of the "result =" part
  \begin{tikzpicture}[remember picture,overlay]%
    \path let
    \p1=(\getBound{cost.south west}),
    \p2=(\getBound{cost.south west2})
    in \pgfextra{
      \tikzmath{\costResultWidth=\x2-\x1;}
      \xdef\costResultWidth{\costResultWidth}
    };
  \end{tikzpicture}%
}

\newcommand{\OPTBG}[4]{%
  \theBackground%
  {\border}%
  {#1}%
  {#2}%
  {#3}%
  {#4}
}

\newcommand{\theProblem}[8]{
  % #1: \result
  % #2: \constrained
  % #3: \task
  % #4: \variables
  % #5: \objective
  % #6: list of constraints
  % #7: \@label
  % #8: \@caption
  % Print the optimization problem
  \OPTBG{cost}{\printCost[false]{#1}{#2}{#3}{#4}{#5}}{0}{#8}%
  \printCost[true]{#1}{#2}{#3}{#4}{#5}%
  \ifthenelse{\equal{#7}{}}{}{\label{eq:#7_a}}
  \ifthenelse{\equal{#2}{true}}{%
    \\%
    \forloop{kk}{0}{\arabic{kk}<\listlen#6[]}{%
      \setcounter{jj}{\value{kk}+1}%
      \setcounter{ll}{\value{kk}+2}%
      &\setBound{constraint\arabic{jj}.south west}%
      \OPTBG{constraint}{#6}{\arabic{jj}}{}%
      #6[\arabic{jj}]%
      \setBound{constraint\arabic{jj}.south east}%
      \ifthenelse{\equal{#7}{}}{}{\label{eq:#7_\alph{ll}}}%
      \ifthenelse{\equal{\arabic{jj}}{\listlen#6[]}}{}{\\}
    }
  }{}
}

\newcommand{\phantomoutline}[1]{
  % \node[outline box] (outline) at (\getBound{cost.south west})
  % {\phantom{$\displaystyle#1$}};
  \node[
  draw=none,
  inner sep=0
  ] (outline) at (0,0)
  {\pgfmark{eqstart\theEqStartCounter}\phantom{$\displaystyle#1$}};
  % \node[
  % draw=black,
  % inner sep=0
  % ] (outline) at (5,0) {\pgfmark{eqstart\theEqStartCounter}{$\displaystyle#1$}};
  \path let
  \p1=(outline.north east),
  \p2=(outline.south east),
  \p3=(pic cs:eqstart\theEqStartCounter)
  in \pgfextra{
    \tikzmath{
      \totalHeight=\y1-\y2;
      \belowShift=\y3-\y2;}
    \xdef\totalHeight{\totalHeight}
    \xdef\belowShift{\belowShift}
  };
  \stepcounter{EqStartCounter}
}

\newcommand{\computeProblemWidth}[2]{
  % #1: \constrained
  % #2: list of constraints
  % Compute the problem width
  \begin{tikzpicture}[remember picture,overlay]
    \path let
    \p1=(\getBound{cost.south west}),
    \p2=(\getBound{cost.south east})
    in \pgfextra{
      \tikzmath{\problemWidth=\x2-\x1;}
      \xdef\problemWidth{\problemWidth}
    };
    \ifthenelse{\equal{#1}{true}}{
      \foreach \i in {1,...,\listlen#2[]} {
        \path let
        \p1=(\getBound{cost.south west}),
        \p2=(\getBound{constraint\i.south east})
        in \pgfextra{
          \tikzmath{\problemWidth=max(\problemWidth,\x2-\x1);}
          \xdef\problemWidth{\problemWidth}
        };
      }
    }{}
  \end{tikzpicture}
}

\define@key{optimusKeys}{task}{\def\task{#1}}%
\define@key{optimusKeys}{objective}{\def\objective{#1}}%
\define@key{optimusKeys}{variables}{\def\variables{#1}}%
\define@key{optimusKeys}{result}{\def\result{#1}}%
\define@key{optimusKeys}{border}{\def\border{#1}}%
\define@key{optimusKeys}{label}{\def\@label{#1}}%
\define@key{optimusKeys}{tag}{\def\@tag{#1}}%
\define@key{optimusKeys}{caption}{\def\@caption{#1}}%
%
\define@key{optimusKeys}{plabel}{\def\@label{#1}}% DEPRECATED, use {label}
\define@key{optimusKeys}{ptag}{\def\@tag{#1}}% DEPRECATED, use {tag}
\define@key{optimusKeys}{pcaption}{\def\@caption{#1}}% DEPRECATED, use {caption}
%
\NewEnviron{optimus}[1][]{%
  % Parameters
  \setkeys{optimusKeys}{plabel=,#1}%
  \setkeys{optimusKeys}{label=,#1}%
  \setkeys{optimusKeys}{border=false,#1}%
  \setkeys{optimusKeys}{task=\min,#1}%
  \setkeys{optimusKeys}{objective=,#1}%
  \setkeys{optimusKeys}{variables=,#1}%
  \setkeys{optimusKeys}{result=,#1}%
  \setkeys{optimusKeys}{ptag=,#1}%
  \setkeys{optimusKeys}{tag=,#1}%
  \setkeys{optimusKeys}{pcaption=,#1}
  \setkeys{optimusKeys}{caption=,#1}
  %
  % Prepare variables
  \stepcounter{ProblemCounter}%
  \ifthenelse{\equal{\BODY}{}}{%
    \xdef\constrained{false}}{%
    \xdef\constrained{true}
    \setsepchar{\#}%
    \readlist\constraints{\BODY}}
  \ifthenelse{\equal{\@caption}{}}{
    \xdef\writeCaption{false}}{
    \xdef\writeCaption{true}}
  % Print the problem
  \computeProblemWidth{\constrained}{\constraints}%
  \newcommand{\printProblemBody}{
    \theProblem%
    {\result}%
    {\constrained}%
    {\task}%
    {\variables}%
    {\objective}%
    {\constraints}%
    {\@label}%
    {\@caption}
  }
  \ifthenelse{\equal{\constrained}{true}}{
    \ifthenelse{\equal{\@label}{}}{
      \begin{align*}
        \printProblemBody
      \end{align*}
    }{
      \ifthenelse{\equal{\@tag}{}}{
        \begin{subequations}
          \label{problem:\@label}
          \begin{align}
            \printProblemBody
          \end{align}
        \end{subequations}
      }{
        \begin{taggedsubequations}{\@tag}
          \label{problem:\@label}
          \begin{align}
            \printProblemBody
          \end{align}
        \end{taggedsubequations}
      }
    }
  }{
    \ifthenelse{\equal{\@label}{}}{
      \begin{equation*}
        \printProblemBody
      \end{equation*}
    }{
      \begin{equation}
        \printProblemBody
      \end{equation}\clonelabel{problem:\@label}{eq:\@label_a}
    }
  }
}

\makeatother

%%% Local Variables:
%%% mode: latex
%%% TeX-master: t
%%% End:


% ..:: Miscellaneous ::..

\newcommand{\PhD}{Ph.D.\xspace}
\newcommand{\Behcet}{Beh\c{c}et}
\newcommand{\Acikmese}{A\c{c}{\i}kme\c{s}e}
\newcommand*\samethanks[1][\value{footnote}]{\footnotemark[#1]}
\makeatletter
\@ifclassloaded{beamer}{}{\newcommand{\alert}[1]{\textbf{#1}}}
\makeatother
\newcommand{\lcvx}{LCvx\xspace}
\newcommand{\emphasize}[1]{\textit{#1}}
\newcommand{\scvx}{SCvx\xspace}
\newcommand{\textsub}[1]{_{\mathrm{#1}}}
\newcommand{\defintext}[1]{\textbf{#1}}
\newenvironment{qanswer}{\color{blue}}{}
\newenvironment{unfinished}{\color{red}\itshape}{}
\newcommand{\makeunderscoreletter}{\catcode`\_11}
\newcommand{\makeunderscoreother}{\catcode`\_8}
% Referencing low-level code
\def\SectionLabel{section}
\def\SubsectionLabel{subsection}
\def\SubsubsectionLabel{subsubsection}
\def\ChapterLabel{chapter}
\def\AppendixLabel{appendix}
\def\FigureLabel{fig}
\def\TableLabel{table}
\def\FootnoteLabel{footnote}
\def\AlgorithmLabel{alg}
\def\ProblemLabel{problem}
\def\PropertyLabel{property}
\def\TheoremLabel{theorem}
\def\CorollaryLabel{corollary}
\def\ConditionLabel{condition}
\def\AssumptionLabel{assumption}
\def\PropositionLabel{proposition}
\def\DefinitionLabel{definition}
\def\LemmaLabel{lemma}
\def\ExampleLabel{example}
\def\ListingLabel{listing}
\newcommand{\GetLabel}[1]{\expandafter\csname #1Label\endcsname}
\makeatletter
\define@key{printRefKeys}{otherLabel}{\def\otherLabel{#1}}
\define@key{printRefKeys}{concatenate}{\def\concatenate{#1}}
\makeatother
\newcommand{\PrintRefs}[3][]{%
  % \ifthenelse{{#1}{}}{}{\edef\MyModifiedLabel{#1}}%
  \setkeys{printRefKeys}{otherLabel=,#1}%
  \setkeys{printRefKeys}{concatenate=false,#1}%
  % #2: Element
  % #3: label
  \xdef\MyModifiedLabel{#2}%
  \ifthenelse{\equal{\otherLabel}{}}{}{\xdef\MyModifiedLabel{\otherLabel}}%
  \xdef\MyRefCount{0}%
  \foreach \i in {#3} {%
    \tikzmath{\MyRefCount=int(\MyRefCount+1);}%
    \xdef\MyRefCount{\MyRefCount}%
  }%
  \ifthenelse{\equal{\MyRefCount}{1}}{%
    #2~\ref{\GetLabel{\MyModifiedLabel}:#3}%
  }{%
    \xdef\MyCounter{0}%
    \ifthenelse{\equal{#2}{Corollary}}{%
      Corollaries%
    }{%
      #2s%
    }~%
    \foreach \AlgRef in {#3} {%
      \tikzmath{\MyCounter=int(\MyCounter+1);}%
      \xdef\MyCounter{\MyCounter}%
      \ifthenelse{\equal{\concatenate}{true}}{%
        \pgfmathparse{\MyCounter==1 ? 1 : 0}%
        \ifthenelse{\pgfmathresult>0}{%
          \ref{\GetLabel{\MyModifiedLabel}:\AlgRef}-%
        }{%
          \pgfmathparse{\MyCounter==\MyRefCount ? 1 : 0}%
          \ifthenelse{\pgfmathresult>0}{%
            \ref{\GetLabel{\MyModifiedLabel}:\AlgRef}%
          }{}%
        }%
      }{%
        \pgfmathparse{\MyCounter<\MyRefCount-1 ? 1 : 0}%
        \ifthenelse{\pgfmathresult>0}{%
          \ref{\GetLabel{\MyModifiedLabel}:\AlgRef},~%
        }{%
          \pgfmathparse{\MyCounter<\MyRefCount ? 1 : 0}%
          \ifthenelse{\pgfmathresult>0}{%
            \ref{\GetLabel{\MyModifiedLabel}:\AlgRef}~and~%
          }{%
            \ref{\GetLabel{\MyModifiedLabel}:\AlgRef}%
          }%
        }%
      }%
    }%
  }%
}
% Document section references
\newcommand{\sref}[1]{\PrintRefs{Section}{#1}}
\newcommand{\ssref}[2][]{\PrintRefs[otherLabel=Subsection,#1]{Section}{#2}}
\newcommand{\sssref}[2][]{\PrintRefs[otherLabel=Subsubsection,#1]{Section}{#2}}
\newcommand{\cref}[1]{\PrintRefs{Chapter}{#1}}
\newcommand{\appref}[1]{\PrintRefs{Appendix}{#1}}
\newcommand{\figref}[1]{\PrintRefs{Figure}{#1}}
\newcommand{\tabref}[2][]{\PrintRefs[#1]{Table}{#2}}
\renewcommand{\footref}[1]{\PrintRefs{Footnote}{#1}}
\makeatletter
\define@key{algKeys}{start}{\def\startline{#1}}
\define@key{algKeys}{end}{\def\endline{#1}}
\define@key{algKeys}{show}{\def\showalg{#1}}
\makeatother
\renewcommand{\algref}[2][]{%
  % Parameters%
  \setkeys{algKeys}{start=,#1}%
  \setkeys{algKeys}{end=,#1}%
  \setkeys{algKeys}{show=true,#1}%
  % Print%
  \ifthenelse{\equal{\showalg}{true}}{\PrintRefs{Algorithm}{#2}}{}%
  \ifthenelse{\equal{\startline}{}}{}{~L\ref{alg:#2:line:\startline}%
    \ifthenelse{\equal{\endline}{}}{}{-\ref{alg:#2:line:\endline}}%
  }%
}
% Mathematical element references
\newcommand{\pref}[2][]{\PrintRefs[#1]{Problem}{#2}}
\newcommand{\propref}[2][]{\PrintRefs[#1]{Property}{#2}}
\newcommand{\tref}[2][]{\PrintRefs[#1]{Theorem}{#2}}
\newcommand{\corref}[2][]{\PrintRefs[#1]{Corollary}{#2}}
\newcommand{\conref}[2][]{\PrintRefs[#1]{Condition}{#2}}
\newcommand{\aref}[2][]{\PrintRefs[#1]{Assumption}{#2}}
\newcommand{\prref}[2][]{\PrintRefs[#1]{Proposition}{#2}}
\newcommand{\dref}[2][]{\PrintRefs[#1]{Definition}{#2}}
\newcommand{\lref}[2][]{\PrintRefs[#1]{Lemma}{#2}}
\newcommand{\eref}[2][]{\PrintRefs[#1]{Example}{#2}}
% Other references
\newcommand{\lstref}[2][]{\PrintRefs[#1]{Listing}{#2}}

\newcommand{\figlabel}[1]{\label{fig:#1}}
\newcommand{\tablabel}[1]{\label{table:#1}}

%%% Local Variables:
%%% mode: latex
%%% TeX-master: t
%%% End:
