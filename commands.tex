% Statistics

\DeclareMathOperator{\expectation}{\mathbb{E}}
\newcommand{\EE}{\expectation}
\DeclareMathOperator{\variance}{\mathrm{Var}}
\newcommand{\Var}{\variance}
\newcommand{\Prob}{\mathbb P}

% Sets

\newcommand{\powerSet}[1]{2^{#1}}
\newcommand{\union}{\cup}
\newcommand{\intersection}{\cap}
\newcommand{\Union}{\bigcup}
\newcommand{\Intersection}{\bigcap}
\newcommand{\set}[1]{\mathcal #1}
\newcommand{\setcomplement}[1]{#1^{\mathrm{c}}}
\newcommand{\eucledian}{\mathbb E}
\newcommand{\reals}{\mathbb R}
\newcommand{\extendedreals}{\overline\reals}
\newcommand{\real}{\mathbb R}
\newcommand{\pos}[1][R]{\mathbb #1_{++}}
\newcommand{\nonneg}[1][\reals]{#1_{+}}
\newcommand{\normball}[1][]{%
  % {x : ||x||_#1 <= 1}
  % #1: norm type
  \mathbb B_{#1}}
\newcommand{\integers}{\mathbb Z}
\newcommand{\naturals}{\mathbb N}
\newcommand{\binary}{\mathbb I}
\newcommand{\quaternion}{\mathbb Q}
\newcommand{\symm}{\mathbb S}  % symmetric matrices
\newcommand{\psd}{\mathbb S_{+}}
\newcommand{\pd}{\mathbb S_{++}}
\newcommand{\dom}{\mathrm{dom}}
\newcommand{\epi}{\mathrm{epi}}
% \newcommand{\hypo}{\mathrm{hypo}}
% \newcommand{\interior}[1]{\mathrm{int}(#1)}
\DeclareMathOperator{\interior}{int}
\DeclareMathOperator{\relint}{ri}
\DeclareMathOperator{\aff}{aff}
\DeclareMathOperator{\bd}{bd}
\DeclareMathOperator{\closure}{cl}
\newcommand{\compl}[1]{#1^\mathrm{c}}
\newcommand{\boundary}[1]{\mathrm{bdry}\left(#1\right)}
% \newcommand{\relint}{\mathrm{relint}}
\newcommand{\coni}[1]{\mathrm{coni}(#1)}
\newcommand{\bdry}{\partial}
\newcommand{\image}[1]{\mathrm{im}(#1)}
\DeclareMathOperator{\comain}{dom}
\newcommand{\domain}[1]{\comain(#1)}
\newcommand{\setHtwo}{\set{H}_2}
\newcommand{\setLinfty}{\set{L}_\infty}
\newcommand{\Ellipsoid}[1]{\mathcal E_{#1}}
\newcommand{\sequence}[3][k]{\{#2_{#1}\}_{#1=1}^{#3}}
\newcommand{\minkowplus}{\oplus}
\newcommand{\powerset}[1]{\mathscr{P}(#1)}
\newcommand{\zeroset}{\{0\}}
\DeclareMathOperator{\vol}{vol}
\newcommand{\volume}[1]{\vol\left(#1\right)}
\DeclareMathOperator{\vtcs}{\set V}
\newcommand{\vertices}[1]{\vtcs\left(#1\right)}
\DeclareMathOperator{\myfixedpointset}{fixed}
\newcommand{\fixedpoint}[1]{\myfixedpointset\left(#1\right)}
\DeclareMathOperator{\sublevelsetsymbol}{lev}
\newcommand{\levelset}[2][]{\sublevelsetsymbol_{#1}#2}

% Algorithms

\newcommand{\algvar}[1]{\text{\IfSubStr{#1}{_}{%
    \StrSubstitute{#1}{_}{\textunderscore}}{#1}}}
\newcommand{\true}{\algvar{true}\xspace}
\newcommand{\false}{\algvar{false}\xspace}
\newcommand{\bigO}[1]{\mathcal O\left(#1\right)}

% Analysis

\newcommand{\Lone}[1]{\|#1\|_1}
\newcommand{\Ltwo}[1]{\|#1\|_2}
\newcommand{\FLinf}[2]{\|#1\|_{#2}}
\newcommand{\converge}[1][]{\overset{#1}{\to}}

% Numerical analysis

\renewcommand{\O}[1]{\mathcal O(#1)}

% Calculus

\newcommand{\grad}{\nabla}
\newcommand{\hess}{\nabla^2}
\newcommand{\diff}[2]{\nabla_{#1}#2}
\newcommand{\hot}{\textnormal{h.o.t.}}
\newcommand{\ddx}{\frac{\dd}{\dd x}}

% Convex analysis

\newcommand{\subdiff}[1][none]{%
  \ifthenelse{\equal{#1}{none}}{%
    \partial%
  }{%
    \partial_{#1}%
  }%
}
\newcommand{\normalcone}[2]{\set{N}_{#1}(#2)}
\DeclareMathOperator{\conehull}{cone}
\DeclareMathOperator{\convhull}{conv}
\DeclareMathOperator{\extreme}{ext}
\newcommand{\indicator}[1]{\delta_{#1}}
\newcommand{\support}[2][delta]{\ifthenelse{\equal{#1}{delta}}{%
    \delta^*_{#2}%
  }{\sigma_{#2}}}

% Functions

\newcommand{\heaviside}{\sigma}

% Linear Algebra

\DeclareMathOperator{\nul}{null}
\DeclareMathOperator{\rank}{rank}
\DeclareMathOperator{\range}{range}
\DeclareMathOperator{\linspan}{span}
\newcommand{\diag}[2][short]{\mathrm{diag}\ifthenelse{\equal{#1}{short}}{(#2)}{\left(#2\right)}}
\DeclareMathOperator{\spec}{spec}
\newcommand{\Spec}[1]{\spec\left(#1\right)}
\newcommand{\rpinv}{^\dagger}
\newcommand{\lpinv}{^\ddagger}
\newcommand{\blkdiag}[1]{\mathrm{blkdiag}(#1)}
\DeclareMathOperator{\trace}{trace}
\newcommand{\Tr}[1]{\trace\left(#1\right)}
\newcommand{\svmax}[1]{\bar\sigma(#1)}
\newcommand{\Matrix}[2][]{%
  \ifthenelse{\equal{#1}{}}{}{\setlength\arraycolsep{#1}}%
  \begin{bmatrix}#2\end{bmatrix}}
\newcommand{\Mnl}{\protect\\} % protected newline

% Generic math

\newcommand{\where}{:}
\newcommand{\given}{\mid}
\newcommand{\CC}[1]{\mathcal C^{#1}}
\newcommand{\inv}[1][]{\ifthenelse{\equal{#1}{}}{^{-1}}{^{-#1}}}
\newcommand{\invT}{^{-\transp}}
\newcommand{\invsq}{^{-\frac{1}{2}}}
\newcommand{\polar}{^{\circ}}
\newcommand{\ortho}{^{\perp}}
\newcommand{\powsq}{^{\frac{1}{2}}}
\newcommand{\ceil}[1]{\left\lceil#1\right\rceil}
\newcommand{\floor}[1]{\left\lfloor#1\right\rfloor}
\newcommand{\sign}{\mathrm{sgn}}
\newcommand{\norm}[2][]{\left\|#2\right\|_{#1}}
\newcommand{\dualnorm}[2][]{\norm[#1]{#2}^*}
\newcommand{\vertiii}[1]{{\left\vert\kern-0.25ex\left\vert\kern-0.25ex\left\vert #1 
        \right\vert\kern-0.25ex\right\vert\kern-0.25ex\right\vert}}
\newcommand{\inorm}[2][]{\vertiii{#2}_{#1}}
\newcommand{\seq}[1]{\{#1\}}
\newcommand{\sgn}{\mbox{\textrm{sgn}}}

\makeatletter
\define@key{projKeys}{set}{\def\set{#1}}
\define@key{proxKeys}{t}{\def\t{#1}}
\define@key{proxKeys}{f}{\def\f{#1}}
\makeatother
\DeclareMathOperator{\projecti}{proj}
\DeclareMathOperator{\proximal}{prox}
\newcommand{\Proj}[2][]{%
  \setkeys{projKeys}{set=,#1}%
  \mathchoice{\underset{\set}{\projecti}}%
  {\projecti_{\set}}{\projecti_{\set}}{\projecti_{\set}}%
  \left(#2\right)%
}
\newcommand{\Prox}[2][]{%
  \setkeys{proxKeys}{t=t,#1}%
  \setkeys{proxKeys}{f=f,#1}%
  \mathchoice{\underset{\t\f}{\proximal}}%
  {\proximal_{\t\f}}{\proximal_{\t\f}}{\proximal_{\t\f}}%
  \ifthenelse{\equal{#2}{}}{}{\left(#2\right)}%
}
\DeclareMathOperator{\distance}{dist}
\newcommand{\dist}[2][]{%
  \distance_{#1}\ifthenelse{\equal{#2}{}}{}{(#2)}
}
\newcommand{\gauge}[2][]{%
  \gamma_{#1}\ifthenelse{\equal{#2}{}}{}{(#2)}
}

\newcommand{\Pre}{\mathop\text{Pre}}
\newcommand{\PP}{$\mathcal{P}$\xspace}
\newcommand{\NP}{$\mathcal{NP}$\xspace}
\newcommand{\NPhard}{$\mathcal{NP}$-hard\xspace}
\newcommand{\NPcomplete}{$\mathcal{NP}$-complete\xspace}
\newcommand{\transp}{{\scriptscriptstyle\mathsf{T}}}
\newcommand{\T}{^\transp}
\newcommand{\innerproduct}[3][]{\langle#2,#3\rangle_{#1}}
\newcommand{\inner}[2][]{\langle#2\rangle_{#1}}
\newcommand{\dd}{\mathrm{d}}
\newcommand{\dt}{\dd t}
\newcommand{\fromabove}{\downarrow}
\newcommand{\frombelow}{\uparrow}
\newcommand{\fun}[2][1]{%
  #2(%
  \foreach \index in {1, ..., #1} {%
    \ifthenelse{\equal{\index}{#1}}{%
      \cdot%
    }{%
      \cdot,%
    }%
  })}
\newcommand{\define}[1]{\textit{#1}}%
\newcommand{\definedas}[1][tri]{%
  \ifthenelse{\equal{#1}{tri}}{\triangleq}{\coloneqq}}
\newcommand{\inlineint}{\textstyle\int}
\newcommand{\inlinesum}{\textstyle\sum}
\newcommand{\evalat}[2]{\left.#1\right|_{#2}}
\newcommand{\epsmach}{\epsilon_{mach}}
\renewcommand{\implies}{\Rightarrow}
\renewcommand{\iff}{\Leftrightarrow}
\newcommand{\mleq}{\preceq}
\newcommand{\mgeq}{\succeq}
\newcommand{\mle}{\prec}
\newcommand{\mgr}{\succ}
\DeclareMathOperator*{\exptx}{exp}
\renewcommand{\exp}[2][exponent]{\ifthenelse{\equal{#1}{exponent}}{e^{#2}}{\exptx\left(#2\right)}}
\DeclareMathOperator*{\expsf}{\mathsf{exp}}
\newcommand{\expm}[2][exponent]{\ifthenelse{\equal{#1}{exponent}}{\mathsf{e}^{#2}}{\expsf\left(#2\right)}}
\newcommand{\ttilde}[1]{\tilde{\raisebox{0pt}[0.85\height]{%
      $\tilde{#1}$}}}
\newcommand{\tttilde}[1]{\tilde{\raisebox{0pt}[0.88\height]{%
      $\tilde{\raisebox{0pt}[0.85\height]{%
          $\tilde{#1}$}}$}}}
\newcommand{\overlline}[1]{\overline{\overline{#1}}}
\newcommand{\bbar}[1]{\bar{\bar{#1}}}
\newcommand{\prefix}[2][]{{}^{#1}_{#2}}

\makeatletter
\DeclareFontFamily{U}{tipa}{}
\DeclareFontShape{U}{tipa}{m}{n}{<->tipa10}{}
\newcommand{\arc@char}{{\usefont{U}{tipa}{m}{n}\symbol{62}}}%
\renewcommand{\arc}[1]{\mathpalette\arc@arc{#1}}
\newcommand{\arc@arc}[2]{%
  \sbox0{$\m@th#1#2$}%
  \vbox{
    \hbox{\resizebox{\wd0}{\height}{\arc@char}}
    \nointerlineskip
    \box0
  }%
}
\makeatother

\makeatletter
\newenvironment{taggedsubequations}[1]
{%
  % \end{subequations} will advance `equation`
  \addtocounter{equation}{-1}%
  \begin{subequations}%
    % set the current label
    \def\@currentlabel{#1}%
    % redefine \theequation
    \renewcommand{\theequation}{#1.\alph{equation}}%
  }
  {\end{subequations}}
\makeatother %not \makeatletter

% Geometry

\newcommand{\fr}[1]{\mathcal{F}_{\mathcal{#1}}}

% Measure theory

\newcommand{\alev}[1]{\text{a.e. }#1}
\newcommand{\gen}[1]{\text{gen. }#1}
\newcommand{\ev}[3][c]{%
  \ifthenelse{\equal{#1}{c}}{%
    \ifthenelse{\equal{#2}{}}{\forall[0,#3]}{\forall[#2,#3]}%
  }{%
    \ifthenelse{\equal{#1}{o}}{%
      \ifthenelse{\equal{#2}{}}{\forall(0,#3)}{\forall(#2,#3)}%
    }{%
      \ifthenelse{\equal{#1}{oc}}{%
        \ifthenelse{\equal{#2}{}}{\forall(0,#3]}{\forall(#2,#3]}%
      }{%
        \ifthenelse{\equal{#2}{}}{\forall[0,#3)}{\forall[#2,#3)}%
      }%
    }%
  }%
}

% Optimization

\DeclareMathOperator*{\argmax}{argmax}
\DeclareMathOperator*{\argmin}{argmin}
\DeclareMathOperator*{\minimize}{\mathrm{minimize}}
\DeclareMathOperator*{\maximize}{\mathrm{maximize}}
\DeclareMathOperator*{\find}{\mathrm{find}}

% Miscellaneous

\newcommand{\PhD}{Ph.D.\xspace}
\newcommand{\Behcet}{Beh\c{c}et}
\newcommand{\Acikmese}{A\c{c}{\i}kme\c{s}e}
\newcommand*\samethanks[1][\value{footnote}]{\footnotemark[#1]}
\makeatletter
\@ifclassloaded{beamer}{}{\newcommand{\alert}[1]{\textbf{#1}}}
\makeatother
\newcommand{\lcvx}{LCvx\xspace}
\newcommand{\emphasize}[1]{\textit{#1}}
\newcommand{\scvx}{SCvx\xspace}
\newcommand{\textsub}[1]{_{\mathrm{#1}}}
\newcommand{\defintext}[1]{\textbf{#1}}
\newenvironment{qanswer}{\color{blue}}{}
\newenvironment{unfinished}{\color{red}\itshape}{}
\newcommand{\makeunderscoreletter}{\catcode`\_11}
\newcommand{\makeunderscoreother}{\catcode`\_8}
% Referencing low-level code
\def\SectionLabel{section}
\def\SubsectionLabel{subsection}
\def\SubsubsectionLabel{subsubsection}
\def\ChapterLabel{chapter}
\def\AppendixLabel{appendix}
\def\FigureLabel{fig}
\def\TableLabel{table}
\def\FootnoteLabel{footnote}
\def\AlgorithmLabel{alg}
\def\ProblemLabel{problem}
\def\PropertyLabel{property}
\def\TheoremLabel{theorem}
\def\CorollaryLabel{corollary}
\def\ConditionLabel{condition}
\def\AssumptionLabel{assumption}
\def\PropositionLabel{proposition}
\def\DefinitionLabel{definition}
\def\LemmaLabel{lemma}
\def\ExampleLabel{example}
\newcommand{\GetLabel}[1]{\expandafter\csname #1Label\endcsname}
\makeatletter
\define@key{printRefKeys}{otherLabel}{\def\otherLabel{#1}}
\define@key{printRefKeys}{concatenate}{\def\concatenate{#1}}
\makeatother
\newcommand{\PrintRefs}[3][]{%
  % \ifthenelse{{#1}{}}{}{\edef\MyModifiedLabel{#1}}%
  \setkeys{printRefKeys}{otherLabel=,#1}%
  \setkeys{printRefKeys}{concatenate=false,#1}%
  % #2: Element
  % #3: label
  \xdef\MyModifiedLabel{#2}%
  \ifthenelse{\equal{\otherLabel}{}}{}{\xdef\MyModifiedLabel{\otherLabel}}%
  \xdef\MyRefCount{0}%
  \foreach \i in {#3} {%
    \tikzmath{\MyRefCount=int(\MyRefCount+1);}%
    \xdef\MyRefCount{\MyRefCount}%
  }%
  \ifthenelse{\equal{\MyRefCount}{1}}{%
    #2~\ref{\GetLabel{\MyModifiedLabel}:#3}%
  }{%
    \xdef\MyCounter{0}%
    #2s~%
    \foreach \AlgRef in {#3} {%
      \tikzmath{\MyCounter=int(\MyCounter+1);}%
      \xdef\MyCounter{\MyCounter}%
      \ifthenelse{\equal{\concatenate}{true}}{%
        % \pgfmathparse{\MyCounter==1 ? 1 : 0}%
        % \ifthenelse{\pgfmathresult>0}{%
        %   \ref{\GetLabel{\MyModifiedLabel}:\AlgRef}-%
        % }{%
        %   % \pgfmathparse{\MyCounter==\myRefCount ? 1 : 0}%
        %   \ifthenelse{\equal{\MyCounter}{\myRefCount}}{%
        %     \ref{\GetLabel{\MyModifiedLabel}:\AlgRef}%
        %   }{}%
        %   }%
        \pgfmathparse{\MyCounter==1 ? 1 : 0}%
        \ifthenelse{\pgfmathresult>0}{%
          \ref{\GetLabel{\MyModifiedLabel}:\AlgRef}-%
        }{%
          \pgfmathparse{\MyCounter==\MyRefCount ? 1 : 0}%
          \ifthenelse{\pgfmathresult>0}{%
            \ref{\GetLabel{\MyModifiedLabel}:\AlgRef}%
          }{}%
        }%
      }{%
        \pgfmathparse{\MyCounter<\MyRefCount-1 ? 1 : 0}%
        \ifthenelse{\pgfmathresult>0}{%
          \ref{\GetLabel{\MyModifiedLabel}:\AlgRef},~%
        }{%
          \pgfmathparse{\MyCounter<\MyRefCount ? 1 : 0}%
          \ifthenelse{\pgfmathresult>0}{%
            \ref{\GetLabel{\MyModifiedLabel}:\AlgRef}~and~%
          }{%
            \ref{\GetLabel{\MyModifiedLabel}:\AlgRef}%
          }%
        }%
      }%
    }%
  }%  
}
% Document section references
\newcommand{\sref}[1]{\PrintRefs{Section}{#1}}
\newcommand{\ssref}[2][]{\PrintRefs[otherLabel=Subsection,#1]{Section}{#2}}
\newcommand{\sssref}[2][]{\PrintRefs[otherLabel=Subsubsection,#1]{Section}{#2}}
\newcommand{\cref}[1]{\PrintRefs{Chapter}{#1}}
\newcommand{\appref}[1]{\PrintRefs{Appendix}{#1}}
\newcommand{\figref}[1]{\PrintRefs{Figure}{#1}}
\newcommand{\tabref}[2][]{\PrintRefs[#1]{Table}{#2}}
\renewcommand{\footref}[1]{\PrintRefs{Footnote}{#1}}
\makeatletter
\define@key{algKeys}{start}{\def\startline{#1}}
\define@key{algKeys}{end}{\def\endline{#1}}
\define@key{algKeys}{show}{\def\showalg{#1}}
\makeatother
\renewcommand{\algref}[2][]{%
  % Parameters%
  \setkeys{algKeys}{start=,#1}%
  \setkeys{algKeys}{end=,#1}%
  \setkeys{algKeys}{show=true,#1}%
  % Print%
  \ifthenelse{\equal{\showalg}{true}}{\PrintRefs{Algorithm}{#2}}{}%
  \ifthenelse{\equal{\startline}{}}{}{~L\ref{alg:#2:line:\startline}%
    \ifthenelse{\equal{\endline}{}}{}{-\ref{alg:#2:line:\endline}}%
  }%
}
% Mathematical element references
\newcommand{\pref}[2][]{\PrintRefs[#1]{Problem}{#2}}
\newcommand{\propref}[2][]{\PrintRefs[#1]{Property}{#2}}
\newcommand{\tref}[2][]{\PrintRefs[#1]{Theorem}{#2}}
\newcommand{\corref}[2][]{\PrintRefs[#1]{Corollary}{#2}}
\newcommand{\conref}[2][]{\PrintRefs[#1]{Condition}{#2}}
\newcommand{\aref}[2][]{\PrintRefs[#1]{Assumption}{#2}}
\newcommand{\prref}[2][]{\PrintRefs[#1]{Proposition}{#2}}
\newcommand{\dref}[2][]{\PrintRefs[#1]{Definition}{#2}}
\newcommand{\lref}[2][]{\PrintRefs[#1]{Lemma}{#2}}
\newcommand{\eref}[2][]{\PrintRefs[#1]{Example}{#2}}

% % Tables

% \makeatletter

% \define@key{@table@keys}{type}{\def\@type{#1}}
% \define@key{@table@keys}{columns}{\def\@columns{#1}}
% \newcommand{\customcell}[2][]{%
%   %
%   % Changes particular cell to type of cell using \multicolumn.
%   %
%   % Inputs
%   % ------
%   % #1 (optional): parameters
%   % #2: cell text
%   \setkeys{@table@keys}{type=c,#1}%
%   \setkeys{@table@keys}{columns=1,#1}%
%   \multicolumn{\@columns}{\@type}{#2}%
% }

% \makeatother

%%% Local Variables:
%%% mode: latex
%%% TeX-master: t
%%% End:
