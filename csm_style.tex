\makeatletter
%%%%%%%%%%%%%%%%%%%%%%%%%%%%%%%%%%%%%%%%%%%%%%%%%%%%%%%%%%%%%%%%%%%%%%%%%%%%%%%%
% ..:: Fonts ::..
% See fonts at: https://www.tug.org/pracjourn/2006-1/hartke/hartke.pdf
% \usepackage{mathpple}
% \usepackage{pxfonts}
%%%%%%%%%%%%%%%%%%%%%%%%%%%%%%%%%%%%%%%%%%%%%%%%%%%%%%%%%%%%%%%%%%%%%%%%%%%%%%%%
% ..:: Initialization function which must be called ::..
\def\csm@refs{} % {label:reftext} associations
\newcommand{\csm@load@ref}{%
  \pykzmathinline[@refs@cache@exists]{'true' if os.path.exists('../../refs.txt')%
    else 'false'}%
  \ifthenelse{\boolean{\@refs@cache@exists}}{%
    \pykzexec{f=open('../../refs_cache.tex','w');
      f.write(chr(92)+'xdef'+chr(92)+'out'+chr(123));
      f.write(open('../../refs.txt','r').read());
      f.write(chr(125));
      f.close()}%
    \input{refs_cache.tex}%
    \xdef\csm@refs{\out}
  }{%
    \xdef\csm@refs{}%
  }%
}
\newcommand{\csm@add@ref}[1]{%
  \ifthenelse{\equal{\csm@refs}{}}{%
    \xdef\csm@refs{'\@key@label':#1}%
  }{%
    \pykzexec{reflist=eval('{'+open('../../refs.txt','r').read()+'}')}%
    \pykzmathinline[@label@exists]{'true' if '\@key@label' in reflist.keys()%
      else 'false'}%
    \ifthenelse{\boolean{\@label@exists}}{}{%
      \xdef\csm@refs{\csm@refs,'\@key@label':#1}
    }%
  }%
\begin{@pie@shell}
printf "\csm@refs" > refs.txt
\end{@pie@shell}%
}
\newcommand{\csm@init@bib}{%
\begin{@pie@shell}
printf '[]' > main_bib.txt &&
printf '[]' > sb_bib.txt
\end{@pie@shell}%
}
\newcommand{\csminit}{
  \csm@load@ref{}
  \csm@init@bib{}
}
%%%%%%%%%%%%%%%%%%%%%%%%%%%%%%%%%%%%%%%%%%%%%%%%%%%%%%%%%%%%%%%%%%%%%%%%%%%%%%%% 
% ..:: Page geometry ::..
\usepackage{geometry}
\def\side@margin{1.5cm}
\def\top@margin{2.2cm}
\def\bot@margin{1.6cm}
\def\foot@margin{6mm}
\def\column@sep{3mm}
\geometry{
  papersize={200mm,273mm},
  left=\side@margin,
  right=\side@margin,
  top=\top@margin,
  bottom=\bot@margin,
  footskip=\foot@margin,
  twocolumn,
  columnsep=\column@sep
}
%%%%%%%%%%%%%%%%%%%%%%%%%%%%%%%%%%%%%%%%%%%%%%%%%%%%%%%%%%%%%%%%%%%%%%%%%%%%%%%%
% ..:: Section titles ::..
\usepackage{helvet}
\titleformat{\section}{\normalfont\fontsize{10}{12}\sffamily\bfseries\uppercase}{}{0cm}{#1}
\titlespacing*{\section}{0cm}{4mm}{0mm}
\titleformat{\subsection}{\normalfont\fontsize{10}{12}\sffamily\bfseries\slshape}{}{0cm}{#1}
\titlespacing*{\subsection}{0cm}{4mm}{0mm}
%%%%%%%%%%%%%%%%%%%%%%%%%%%%%%%%%%%%%%%%%%%%%%%%%%%%%%%%%%%%%%%%%%%%%%%%%%%%%%%%
% ..:: Header/footer ::..
\def\csm@pubdate{}
\newcommand{\csmdate}[1]{\xdef\csm@pubdate{#1}}
\usepackage{fontawesome}
\fancypagestyle{csmfancy}
{
  \fancyhf{}
  \renewcommand{\headrulewidth}{0pt}
  \fancyfoot[LO]{
    \sffamily
    \fontsize{9}{11}
    \textbf{\thepage}~~%
    \fontsize{6}{8}
    \uppercase{\bfseries IEEE Control Systems}~~%
    \tikz[outer sep=0,inner sep=0]{%
      \node[scale=0.8] at (0,0) {\faChevronRight};
      \node[scale=0.8] at (0.4em,0) {\faChevronRight};}
    {\normalfont\sffamily\MakeUppercase\csm@pubdate}
  }%
  \fancyfoot[RE]{
    \fontsize{6}{8}
    \sffamily
    {\MakeUppercase\csm@pubdate}
    \tikz[outer sep=0,inner sep=0]{
      \node[scale=0.8] at (0,0) {\faChevronLeft};
      \node[scale=0.8] at (0.4em,0) {\faChevronLeft};}
    \uppercase{\bfseries IEEE Control Systems}~~%
    \fontsize{9}{11}
    \textbf{\thepage}%
  }%
}
\pagestyle{csmfancy}
%%%%%%%%%%%%%%%%%%%%%%%%%%%%%%%%%%%%%%%%%%%%%%%%%%%%%%%%%%%%%%%%%%%%%%%%%%%%%%%%
% ..:: Equations ::..
\allowdisplaybreaks
%%%%%%%%%%%%%%%%%%%%%%%%%%%%%%%%%%%%%%%%%%%%%%%%%%%%%%%%%%%%%%%%%%%%%%%%%%%%%%%%
% ..:: Bibliography ::..
\renewcommand*{\bibfont}{\scriptsize}
\setlength{\bibsep}{1pt}
\renewcommand{\bibsection}{}
\setlength{\bibhang}{0pt}
\nobibliography*
% >> Customized citation for main body/sidebar distinguishing <<
\gdef\citations@sb@count{0}
\let\origcite\cite
\renewcommand{\cite}[2][]{%
  \pykzmathinline[num@cites]{len('#2'.split(','))}%
  \gdef\cite@count{0}%
  \foreach \@citation in {#2} {%
    \counter@increment{\cite@count}%
    % >> Check if citation is in "main" list <<
    \pykzexec{main_bib=eval(open('../../main_bib.txt','r').read())}%
    \pykzmathinline[citation@number]{next((i for i,x in
      enumerate(main_bib) if x=='\@citation'),-2)+1}%
    \pykzmathinline[citation@in@main]{'true' if \citation@number>-1 else 'false'}%
    \ifthenelse{\boolean{\citation@in@main}}{%
      \csm@cite[#1]{\@citation}{\citation@number}%
    }{}%
    % >> Check if citation is in "sidebar" list <<
    \ifthenelse{\not\boolean{\citation@in@main}}{%
      \pykzexec{sb_bib=eval(open('../../sb_bib.txt','r').read())}%
      \pykzmathinline[citation@number]{next((i for i,x in
        enumerate(sb_bib) if x=='\@citation'),-2)+1}%
      \pykzmathinline[citation@in@sb]{'true' if \citation@number>-1 else 'false'}%
      \ifthenelse{\boolean{\citation@in@sb}}{%
        \csm@cite[#1]{\@citation}{S\citation@number}%
      }{}%
    }{%
      \gdef\citation@in@sb{false}%
    }%
    % >> Add citation to list, if does not yet exist <<
    \pykzmathinline[citation@exists]{'true' if ('\citation@in@main'=='true'%
      or '\citation@in@sb'=='true') else 'false'}%
    \ifthenelse{\boolean{\citation@exists}}{}{%
      \ifthenelse{\not\boolean{\@inside@sb}}{%
        \pykzexec{main_bib=main_bib+['\@citation']}%
        \pykzmathinline[citations@main@count]{len(main_bib)}%
        \csm@cite[#1]{\@citation}{\citations@main@count}%
        % >> Write to file <<
        \pykzexec{f=open('../../main_bib.txt','w');f.write(str(main_bib));f.close()}%
      }{%
        \pykzexec{sb_bib=sb_bib+['\@citation']}%
        \pykzmathinline[citations@sb@count]{len(sb_bib)}%
        \csm@cite[#1]{\@citation}{S\citations@sb@count}%
        % >> Write to file <<
        \pykzexec{f=open('../../sb_bib.txt','w');f.write(str(sb_bib));f.close()}%
      }%
    }%
    \pgfmathparse{\cite@count<\num@cites ? 1:0}%
    \ifthenelse{\pgfmathresult>0}{%
      , %
    }{}%
  }%
}
\newcommand{\csm@cite}[3][]{[\hyperlink{#2}{#3}]}
\newcommand{\csmbibliography}[2][]{%
  \scriptsize%
  \ifthenelse{\equal{#2}{main}}{%
    \def\citation@list{main_bib.txt}
  }{%
    \def\citation@list{sb_bib.txt}
  }%
  \pykzexec{bib=eval(open('../../\citation@list','r').read())}%
  \ifthenelse{\equal{#1}{}}{%
    \pykzmathinline[@bibitems]{len(bib)}%
    \xdef\bib@range{1,...,\@bibitems}%
  }{%
    \xdef\bib@range{#1}%
  }%
  \foreach \@i in \bib@range {%
    \pykzmathinline[@citation]{bib[\@i-1]}%
    \ifthenelse{\boolean{\@inside@sb}}{%
      \noindent [S\hypertarget{\@citation}{{\@i}}]~~\bibentry{\@citation}.%
    }{%
      \noindent [\hypertarget{\@citation}{{\@i}}]~~\bibentry{\@citation}.%
    }%
    
  }%
}
%%%%%%%%%%%%%%%%%%%%%%%%%%%%%%%%%%%%%%%%%%%%%%%%%%%%%%%%%%%%%%%%%%%%%%%%%%%%%%%%
% ..:: Figures ::..
% >> Re-define figure environment to expand its optional argument first <<
% Src: https://tex.stackexchange.com/questions/11336/macro-for-figure-position
\let\origfigure\figure
\let\endorigfigure\endfigure
\renewenvironment{figure}[1][]{%
  \expandafter\origfigure\expandafter[#1]%
}{%
  \endorigfigure
}
% >> CSM-aesthetic figure <<
\newcounter{csmsubfigure}
\gdef\@csm@caption@vsep{2mm}
\gdef\@csm@figure@vpad{4mm}
\gdef\@inside@figure{false}
\definecolor{csm@figure@text}{HTML}{186cb6}
\definecolor{csm@figure@bg}{HTML}{dde4f3}
\define@key{@csm@keys}{caption}{\def\@key@caption{#1}}
\define@key{@csm@keys}{label}{\def\@key@label{#1}}
\define@key{@csm@keys}{columns}{\def\@key@columns{#1}}
\define@key{@csm@keys}{position}{\def\@key@position{#1}}
\newenvironment{csmfigure}[1][]
{%
  \setkeys{@csm@keys}{caption=,#1}
  \setkeys{@csm@keys}{label=,#1}
  \setkeys{@csm@keys}{columns=1,#1}
  \setkeys{@csm@keys}{position=,#1}
  %
  \ifthenelse{\equal{\@key@columns}{1}}{
    \def\@csm@fig@baselinewidth{\columnwidth}
  }{
    \def\@csm@fig@baselinewidth{\textwidth}
  }
  %
  \setcounter{csmsubfigure}{0}%
  \refstepcounter{figure}
  %
  \xdef\@this@fig@count{\@content@box@counter}
  \begin{contentbox}
    \begin{minipage}{\@csm@fig@baselinewidth}
      \centering
    }
    % BODY
    {
    \end{minipage}
  \end{contentbox}
  %
  \select@figure[\@key@position]{begin}{\@key@columns}
  \pgfdeclarelayer{bg}
  \pgfsetlayers{bg,main}
  \begin{pykzpicture}
    \tikzset{
      csm figure/.append style={
        text width=\@csm@fig@baselinewidth,
        align=center
      },
      csm caption/.append style={
        text width=\@csm@fig@baselinewidth,
        align=left,
        anchor=north west,
        font={\sffamily\small},
        yshift={-\@csm@figure@vpad-\@csm@caption@vsep}
      },
      csm figure background/.append style={
        csm@figure@bg,
        rounded corners=1mm
      }
    }
    % >> Figure picture <<
    \node[csm figure] (fig) at (0,0) {\usecontent[\@this@fig@count]};
    % >> Background <<
    \begin{pgfonlayer}{bg}
      \fill[csm figure background]
      ($(fig.south west)+(0,-\@csm@figure@vpad)$)
      rectangle
      ($(fig.north east)+(0,\@csm@figure@vpad)$);
    \end{pgfonlayer}
    % >> Caption <<
    \node[csm caption] at (fig.south west)
    {{\bfseries\color{csm@figure@text} FIGURE \thefigure}~\@key@caption};
    \label{\@key@label}
    \csm@add@ref{'\thefigure'}
  \end{pykzpicture}
  \select@figure{end}{\@key@columns}
}
\newcommand{\select@figure}[3][]{%
  \ifthenelse{\equal{#2}{begin}}{%
    \ifthenelse{\equal{#3}{1}}{\begin{figure}[#1]}{%
        \ifthenelse{\equal{#1}{t}}{\begin{figure*}[t]}{%
            \ifthenelse{\equal{#1}{b}}{\begin{figure*}[b]}{\begin{figure*}}}}%
        }{\ifthenelse{\equal{#3}{1}}{\end{figure}}{\end{figure*}}}%
}
\define@key{@csm@keys}{width}{\def\@key@width{#1}}
\newenvironment{csmsubfigure}[1][]
{%
  \setkeys{@csm@keys}{width=\textwidth,#1}%
  \setkeys{@csm@keys}{caption=,#1}%
  \setkeys{@csm@keys}{label=,#1}%
  %
  \begin{contentbox}
    \begin{minipage}[t]{\@key@width}
      \centering
    }
    % BODY
    {
    \end{minipage}%
  \end{contentbox}%
  \begin{pykzpicture}
    \tikzset{
      csm caption/.append style={
        text width=\@key@width,
        align=center,
        anchor=north west,
        font={\sffamily\footnotesize},
        yshift=-\@csm@caption@vsep
      }
    }
    % >> Figure picture <<
    \node (fig) at (0,0) {\usecontent};
    % >> Caption <<
    \refstepcounter{csmsubfigure}
    \node[csm caption] at (fig.south west) {(\alph{csmsubfigure})~\@key@caption};
    \label{\@key@label}
    \csm@add@ref{'\thefigure(\alph{csmsubfigure})'}
  \end{pykzpicture}%
}
\newcommand\SLASH{\textbackslash}
\newcommand{\csmfigref}[1]{%
  \pykzexec{reflist=eval('{'+open('../../refs.txt','r').read()+'}')}%
  \pykzmathinline[@fig@reftext]{reflist['fig:#1']}%
  Figure~\hyperref[fig:#1]{\@fig@reftext}%
}
%%%%%%%%%%%%%%%%%%%%%%%%%%%%%%%%%%%%%%%%%%%%%%%%%%%%%%%%%%%%%%%%%%%%%%%%%%%%%%%%
% ..:: Sidebars ::..
\newcounter{csmsidebar}
\newcounter{csmsidebarequation}
\newcounter{csmsidebarfigure}
\setcounter{csmsidebar}{0}
\setcounter{csmsidebarequation}{0}
\setcounter{csmsidebarfigure}{0}
\definecolor{csm@sidebar@bg}{HTML}{f2e9c4}
\gdef\@inside@sb{false}
\gdef\@csm@sidebar@pad{3mm}
\gdef\@csm@sidebar@title@vsep{1mm}
\define@key{@csm@keys}{title}{\def\@key@title{#1}}
\define@key{@csm@keys}{banner}{\def\@key@banner{#1}}
\newenvironment{csmsidebar}[1][]
{%
  \setkeys{@csm@keys}{title=,#1}
  \setkeys{@csm@keys}{label=,#1}
  \setkeys{@csm@keys}{columns=2,#1} % 1|2|4
  \setkeys{@csm@keys}{banner=,#1} % 1|2|4
  \setkeys{@csm@keys}{position=t,#1} % 1|2|4
  %
  \tikzmath{\@pages=int(max(1,floor(\@key@columns/2)));}
  \ifthenelse{\equal{\@key@columns}{1}}{%
    \def\@csm@fig@baselinewidth{\columnwidth}
  }{
    \tikzmath{\@csm@fig@baselinewidth=\@pages*\textwidth;}
    \dimensionalize{\@csm@fig@baselinewidth}
  }
  %
  \refstepcounter{csmsidebar}
  \xdef\@this@sidebar@count{\@content@box@counter}
  \begin{contentbox}%
    \tikzmath{%
      \@cols=int(\@key@columns);%
      \page@width=\@csm@fig@baselinewidth-2*\@csm@sidebar@pad;%
    }%
    \dimensionalize{\page@width}%
    % >> Set equation numbering to sidebar equations <<
    \xdef\cur@eq@count{\arabic{equation}}%
    \renewcommand{\theequation}{\sffamily S\arabic{equation}}%
    \setcounter{equation}{\arabic{csmsidebarequation}}%
    % >> Set figure numbering to sidebar figures <<
    \xdef\cur@fig@count{\arabic{figure}}%
    \renewcommand{\thefigure}{S\arabic{figure}}%
    \setcounter{figure}{\arabic{csmsidebarfigure}}%
    %
    \gdef\@inside@sb{true}%
    \xdef\sb@citation@start{\citations@sb@count}%
    %
    \xdef\currentpagewidth{\page@width}%
    \begin{minipage}{\page@width}%
      \fontsize{9}{13}\sffamily\sansmath%
      \ifthenelse{\equal{\@cols}{1}}{%
        \@key@banner%
        
      }{%
        \begin{multicols}{\@cols}[\@key@banner]%
        }%
      }%
      % BODY
      {%
        % >> Print references <<
        \xdef\sb@citation@end{\citations@sb@count}%
        \pgfmathparse{\sb@citation@end-\sb@citation@start>0 ? 1:0}%
        \ifthenelse{\pgfmathresult>0}{%
          \tikzmath{\sb@citation@start=int(\sb@citation@start+1);}%
          \section{References}%
          
          \csmbibliography[\sb@citation@start,...,\sb@citation@end]{sb}%
        }{}%
        %
      \ifthenelse{\equal{\@cols}{1}}{}{\end{multicols}}%
      % >> Set equation, figure numbering back to global <<
      \setcounter{csmsidebarequation}{\arabic{equation}}%
      \setcounter{equation}{\cur@eq@count}%
      \setcounter{csmsidebarfigure}{\arabic{figure}}%
      \setcounter{figure}{\cur@fig@count}%
      %
      \gdef\@inside@sb{false}
      %
    \end{minipage}%
  \end{contentbox}
  %
  \def\sb@corner@fillet{1mm}
  \tikzset{
    csm sidebar content/.append style={
      text width=\@csm@fig@baselinewidth,
      align=center,
      % fill=red,
      % fill opacity=0.2,
      anchor=north west
    },
    csm sidebar title/.append style={
      % text width=\@csm@fig@baselinewidth-2*\@csm@sidebar@pad,
      font={\normalfont\fontsize{12}{14}\sffamily\bfseries},
      % fill=yellow,
      xshift=\@csm@sidebar@pad,
      anchor=south west
    },
    csm sidebar background/.append style={
      csm@sidebar@bg,
      rounded corners=\sb@corner@fillet
    }
  }
  %
  \select@figure[\@key@position]{begin}{\@key@columns}
  %
  \pgfdeclarelayer{bg}
  \pgfdeclarelayer{fg}
  \pgfsetlayers{bg,main,fg}
  \begin{pykzpicture}
    % >> Sidebar title <<
    % \begin{pgfonlayer}{fg}
    %   \node[circle,fill=red,minimum size=1mm] at (0,0) {};
    % \end{pgfonlayer}
    \coordinate (origin) at (0,0);
    \node[csm sidebar title] (sb title) at (0,0)
    {\@key@title};
    \label{\@key@label}
    \csm@add@ref{'\@key@title'}
    % >> Sidebar content <<
    \node[csm sidebar content] (sb content) at (0,-\@csm@sidebar@title@vsep)
    {\usecontent[\@this@sidebar@count]};
    \path let \p1=(sb content.north), \p2=(sb content.south east) in \pgfextra{
      \tikzmath{
        \@sb@content@width=\x2-\x1;
        \@sb@content@height=\y1-\y2;
      }
      \dimensionalize{\@sb@content@width,\@sb@content@height}
    };
    \pgfmathparse{\@pages>1 ? 1:0}
    \ifthenelse{\pgfmathresult>0}{%
      \fill[csm sidebar background]
      ($(sb content.south)+(0,-\@csm@sidebar@pad)$)
      rectangle ($(sb content.north east)+(0,\@csm@sidebar@pad)$);
    }{}
    % >> Sidebar background <<
    \begin{pgfonlayer}{bg}
      \fill[csm sidebar background]
      ($(sb content.south west)+(0,-\@csm@sidebar@pad)$)
      rectangle
      ($(sb title.north east-|sb content.north east)+(0,\@csm@sidebar@pad)$);
    \end{pgfonlayer}
  \end{pykzpicture}
  % 
  \select@figure{end}{\@key@columns}
  % 
  \pgfmathparse{\@pages>1 ? 1:0}
  \ifthenelse{\pgfmathresult>0}{%
    % 
    \select@figure[\@key@position]{begin}{\@key@columns}
    %
    \pgfdeclarelayer{bg}
    \pgfdeclarelayer{fg}
    \pgfsetlayers{bg,main,fg}
    \begin{pykzpicture}
      % >> Sidebar title <<
      % \begin{pgfonlayer}{fg}
      %   \node[circle,fill=red,minimum size=1mm] at (0,0) {};
      % \end{pgfonlayer}
      \coordinate (origin) at (0,0);
      \node[csm sidebar title] (sb title) at (0,0)
      {\phantom\@key@title};
      % >> Sidebar content <<
      \begin{scope}[local bounding box={sb content}]
        \path[clip] (0,-\@csm@sidebar@title@vsep) rectangle
        ++(\@sb@content@width-\column@sep/2,-\@sb@content@height);
        \node[csm sidebar content,anchor=north] at
        (-\column@sep/2,-\@csm@sidebar@title@vsep)
        {\usecontent[\@this@sidebar@count]};
      \end{scope}
      % \pgfmathparse{\@pages>1 ? 1:0}
      % \ifthenelse{\pgfmathresult>0}{%
      % \fill[csm sidebar background] (sb content.south) rectangle (sb
      % content.north east);
      % }{}
      %   >> Sidebar background <<
      \begin{pgfonlayer}{bg}
        \fill[csm sidebar background]
        ($(sb content.south west)+
        (-\@csm@sidebar@pad,-\@csm@sidebar@pad)$)
        node(sb bg sw){} rectangle
        ($(sb title.north east-|sb content.north east)+
        (\@csm@sidebar@pad,\@csm@sidebar@pad)$)
        node(sb bg ne){};
      \end{pgfonlayer}
      % >> Page margin fill <<
      \begin{scope}[remember picture,overlay]
        \node[circle,fill=red,minimum size=1mm] at (-1cm,0) {};
        \fill[csm sidebar background]
        ($(sb bg sw)+(2*\sb@corner@fillet,0)$) --
        ($(sb bg sw|-sb bg ne)+(2*\sb@corner@fillet,0)$) --
        ++(-2*\side@margin,0) |-cycle;
      \end{scope}
    \end{pykzpicture}
    % 
    \select@figure{end}{\@key@columns}
    % 
  }{}
}
\newcommand{\csmsbref}[1]{%
  \pykzexec{reflist=eval('{'+open('../../refs.txt','r').read()+'}')}%
  \pykzmathinline[@sb@reftext]{reflist['sb:#1']}%
  ``\hyperref[sb:#1]{\@sb@reftext}''%
}
%%%%%%%%%%%%%%%%%%%%%%%%%%%%%%%%%%%%%%%%%%%%%%%%%%%%%%%%%%%%%%%%%%%%%%%%%%%%%%%%
% ..:: Mathematical text environments (theorem, etc.) ::..
\gdef\@csm@thm@kinds{}
\define@key{@csm@keys}{kind}{\def\@key@kind{#1}}
\newenvironment{csmtheorem}[1][]
{%
  \setkeys{@csm@keys}{kind=theorem,#1}%
  \setkeys{@csm@keys}{title=\@key@kind,#1}%
  \pykzmathinline[@key@title]{'\@key@title'[0].upper()+'\@key@title'[1:]}
  %
  \pykzmathinline[kind@known]{'true' if '\@key@kind' in [\@csm@thm@kinds] else%
    'false'}%
  \ifthenelse{\boolean{\kind@known}}{}{%
    \pykzmathinline[@csm@thm@kinds]{str([\@csm@thm@kinds]+['\@key@kind'])[1:-1]}%
    \newcounter{csm@\@key@kind}%
    \setcounter{csm@\@key@kind}{0}%
  }%
  \refstepcounter{csm@\@key@kind}%
  \vspace{4mm}%
  \noindent%
  {\fontsize{10}{12}\sffamily%
    \@key@title~\arabic{csm@\@key@kind}\newline}%
}
% BODY
{}
% >> Specific instances of csmtheorem <<
\newenvironment{theorem}[1][]{\begin{csmtheorem}[kind=theorem,#1]}{\end{csmtheorem}}
\newenvironment{assumption}[1][]{\begin{csmtheorem}[kind=assumption,#1]}{\end{csmtheorem}}
\newenvironment{condition}[1][]{\begin{csmtheorem}[kind=condition,#1]}{\end{csmtheorem}}
\newenvironment{definition}[1][]{\begin{csmtheorem}[kind=definition,#1]}{\end{csmtheorem}}
%%%%%%%%%%%%%%%%%%%%%%%%%%%%%%%%%%%%%%%%%%%%%%%%%%%%%%%%%%%%%%%%%%%%%%%%%%%%%%%%
% ..:: Nomenclature custom sidebars ::..
\newcommand{\csmsymbols}{
  \begin{csmsidebar}[
    title={List of Symbols},
    columns=1,
    position=H
    ]%
    \pykzexec{syms=eval(open('../../syms.txt','r').read())}%
    \pykzmathinline[@num@sym@defs]{len(syms)-1}%
    \pykzexec{f=open('../../sym_table.tex','w');
      print(chr(92)+'begin{tabularx}{'+chr(92)+'textwidth}{lX}',file=f)}%
    \foreach \@i in {0,...,\@num@sym@defs} {%
      \pykzexec{print((syms[\@i][0]+' '+chr(38)+' '+syms[\@i][1]+' '+
        chr(92)*2).replace(':::',chr(36)).replace('@',chr(92)),file=f)}%
    }%
    \pykzexec{print(chr(92)+'end{tabularx}',file=f);f.close()}%    
    \input{sym_table.tex}%
  \end{csmsidebar}%
}
\newcommand{\csmabbreviations}{
  \begin{csmsidebar}[
    title={List of Abbreviations},
    columns=1,
    position=H
    ]%
    \pykzexec{abbrevs=eval(open('../../abbrevs.txt','r').read())}%
    \pykzmathinline[@num@abbrev@defs]{len(abbrevs)-1}%
    \pykzexec{f=open('../../abbrev_table.tex','w');
      print(chr(92)+'begin{tabularx}{'+chr(92)+'textwidth}{lX}',file=f)}%
    \foreach \@i in {0,...,\@num@abbrev@defs} {%
      \pykzexec{print((abbrevs[\@i][0]+' '+chr(38)+' '+abbrevs[\@i][1]+' '+
        chr(92)*2).replace(':::',chr(36)).replace('@',chr(92)),file=f)}%
    }%
    \pykzexec{print(chr(92)+'end{tabularx}',file=f);f.close()}%    
    \input{abbrev_table.tex}%
  \end{csmsidebar}%
}
\newcommand{\defsym}[2]{%
  % Encode $-->::: to not be misread by shell
  \noexpandarg
  \StrSubstitute{#1}{^^24}{^^3a^^3a^^3a}[\sym@name]
  \StrSubstitute{#2}{^^24}{^^3a^^3a^^3a}[\sym@desc]
  \normalexpandarg
  % Encode \-->@
  \replace@BS{\sym@name}{\sym@name}
  \replace@BS{\sym@desc}{\sym@desc}
  %
  \pykzmathinline[@sym@cache@exists]{'true' if os.path.exists('../../syms.txt')%
    else 'false'}%
  \ifthenelse{\boolean{\@sym@cache@exists}}{}{%
\begin{@pie@shell}
printf '[]' > syms.txt
\end{@pie@shell}%
  }
  %
  \pykzexec{syms=eval(open('../../syms.txt','r').read())}%
  \pykzmathinline[@def@idx]{next((i for i,symdef in enumerate(syms)
    if '\sym@name'==symdef[0]),-1)}%
  \ifthenelse{\equal{\@def@idx}{-1}}{%
    \pykzexec{syms.append(('\sym@name','\sym@desc'))}%
  }{%
    \pykzexec{syms[\@def@idx]=('\sym@name','\sym@desc')}%
  }%
  \pykzexec{f=open('../../syms.txt','w');f.write(str(syms));f.close()}%
}
\newcommand{\defabbrev}[2]{%
  % Encode $-->::: to not be misread by shell
  \noexpandarg
  \StrSubstitute{#1}{^^24}{^^3a^^3a^^3a}[\abbrev@name]
  \StrSubstitute{#2}{^^24}{^^3a^^3a^^3a}[\abbrev@desc]
  \normalexpandarg
  % Encode \-->@
  \replace@BS{\abbrev@name}{\abbrev@name}
  \replace@BS{\abbrev@desc}{\abbrev@desc}
  %
  \pykzmathinline[@abbrev@cache@exists]{'true' if os.path.exists('../../abbrevs.txt')%
    else 'false'}%
  \ifthenelse{\boolean{\@abbrev@cache@exists}}{}{%
\begin{@pie@shell}
printf '[]' > abbrevs.txt
\end{@pie@shell}%
  }
  %
  \pykzexec{abbrevs=eval(open('../../abbrevs.txt','r').read())}%
  \pykzmathinline[@def@idx]{next((i for i,abbrevdef in enumerate(abbrevs)
    if '\abbrev@name'==abbrevdef[0]),-1)}%
  \ifthenelse{\equal{\@def@idx}{-1}}{%
    \pykzexec{abbrevs.append(('\abbrev@name','\abbrev@desc'))}%
  }{%
    \pykzexec{abbrevs[\@def@idx]=('\abbrev@name','\abbrev@desc')}%
  }%
  \pykzexec{f=open('../../abbrevs.txt','w');f.write(str(abbrevs));f.close()}%
}
\newcommand{\replace@BS}[2]{{\escapechar=`@
    \xdef#2{\detokenize\expandafter{#1}\@empty}}}
%%%%%%%%%%%%%%%%%%%%%%%%%%%%%%%%%%%%%%%%%%%%%%%%%%%%%%%%%%%%%%%%%%%%%%%%%%%%%%%% 
\makeatother

%%% Local Variables:
%%% mode: latex
%%% TeX-master: "../main"
%%% End:
