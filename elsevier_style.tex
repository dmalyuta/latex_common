% Font
\usepackage[T1]{fontenc}
\usepackage[utf8]{inputenc}
\usepackage{microtype}
\let\acute\undefined
\let\grave\undefined
\let\ddot\undefined
\let\tilde\undefined
\let\bar\undefined
\let\breve\undefined
\let\check\undefined
\let\hat\undefined
\let\dot\undefined
\let\mathring\undefined
\let\wideparen\undefined
\usepackage{fouriernc} % Math and text font
\usepackage{charter} % Override text font

% Margins
\def\elsSideMargin{1.2cm}
\def\elsTopMargin{1.1cm}
\def\elsColSep{9.5mm}
\usepackage[left=\elsSideMargin,
right=\elsSideMargin,
top=\elsTopMargin,
bottom=\elsTopMargin,
columnsep=\elsColSep]{geometry}

% Line skip
\renewcommand{\baselinestretch}{0.9}

%%%%%%%%%%%%%%%%%%%%%%%
% Figures
%%%%%%%%%%%%%%%%%%%%%%%

\setlength{\mathindent}{0pt} % Left-justified math

%%%%%%%%%%%%%%%%%%%%%%%
% Figures
%%%%%%%%%%%%%%%%%%%%%%%

\captionsetup[figure]{labelfont={bf},font=footnotesize,name={Fig.},labelsep=period}

%%%%%%%%%%%%%%%%%%%%%%%
% Heading and footer
%%%%%%%%%%%%%%%%%%%%%%%
\makeatletter
\newcommand{\listget}[3][]{\list@get[#1]{#2}{#3}}
\makeatother
\fancypagestyle{elsevierfancy}
{
  \fancyhf{}
  \renewcommand{\headrulewidth}{0pt}
  \fancyhead[L]{
    \footnotesize
    \thepage
  }
  \fancyhead[C]{
    \itshape
    \footnotesize
    \xdef\elscounter{1}
    \xdef\elsmaxauthor{3}
    \foreach \authorfullnameaffil in \elsauthorlist {%
      \listget[authorfullname]{\authorfullnameaffil}{2}%
      \StrLeft{\authorfullname}{1}[\authornamefirstchar]%
      \authornamefirstchar.~% % first name
      \elsgetlistelement{\authorfullname}{2}% % last name
      \pgfmathparse{\elscounter==\elsmaxauthor ? 1:0}%
      \ifthenelse{\pgfmathresult>0}{~et al.\breakforeach}{}%
      \pgfmathparse{\elscounter<min(\elsmaxauthor,\elsauthorcount)-1 ? 1:0}%
      \ifthenelse{\pgfmathresult>0}{,~}{}%
      \pgfmathparse{\elscounter==min(\elsmaxauthor,\elsauthorcount)-1 ? 1:0}%
      \ifthenelse{\pgfmathresult>0}{~and~}{}%
      \elsaddtocounter{\elscounter}%
    }
    / Annual Reviews in Control
  }
}
\headheight 28pt              %% put this outside
\headsep 10pt                 %% put this outside

\pagestyle{elsevierfancy}

%%%%%%%%%%%%%%%%%%%%%%%
% Section style
%%%%%%%%%%%%%%%%%%%%%%%

\usepackage{indentfirst}
\setlength\parindent{5mm}

\titleformat{\section}{\normalfont\bfseries\normalsize}{}{0cm}{\thesection.~#1}
\titleformat{\subsection}{\normalfont\itshape\normalsize}{}{0cm}{\thesubsection.~#1}
\titleformat{\subsubsection}{\normalfont\itshape\normalsize}{}{0cm}{\thesubsubsection.~#1}
\titlespacing*{\section}{0cm}{8mm}{4mm}
\titlespacing*{\subsection}{0cm}{5mm}{5mm}
\titlespacing*{\subsubsection}{0cm}{5mm}{\lineskip}


%%%%%%%%%%%%%%%%%%%%%%%
% Document head
%%%%%%%%%%%%%%%%%%%%%%%

\makeatletter
\newcommand{\elstitle}[1]{
  \def\@elstitle{#1}
}
\makeatother

\newcommand{\elsappendtolist}[2]{
  % #1: list
  % #2: element to add to end
  \ifthenelse{\equal{#1}{}}{\xdef#1{#2}}{\xdef#1{#1,#2}}
}

\makeatletter%
\newcommand{\elsgetlistelement}[2]{%
  % #1: list
  % #2: element index (1-indexing)
  \xdef\@@counter{1}%
  \foreach \element in #1 {%
    \pgfmathparse{\@@counter==#2 ? 1:0}%
    \ifthenelse{\pgfmathresult>0}{\element\breakforeach}{}%
    \elsaddtocounter{\@@counter}%
  }%
}
\makeatother%

\makeatletter%
\newcommand{\elsaddtocounter}[2][]{%
  \xdef\@increment{1}%
  \ifthenelse{\equal{#1}{}}{}{\xdef\@increment{#1}}%
  \tikzmath{#2=int(#2+\@increment);}%
  \xdef#2{#2}%
}
\makeatother%

\xdef\elsauthorlist{}
\xdef\elsaffiliationlist{}
\xdef\elsauthorcount{0}
\xdef\elsaffiliationcount{0}
\newcommand{\elsauthor}[2]{%
  \elsappendtolist{\elsauthorlist}{{#1,{#2}}}%
  \elsaddtocounter{\elsauthorcount}%
}
\newcommand{\elsaffiliation}[2]{%
  \elsappendtolist{\elsaffiliationlist}{{#1,{#2}}}%
  \elsaddtocounter{\elsaffiliationcount}%
}

\makeatletter
\define@key{@elspaperhead@keys}{received}{\def\@receive@date{#1}}
\define@key{@elspaperhead@keys}{revised}{\def\@revised@date{#1}}
\define@key{@elspaperhead@keys}{accepted}{\def\@accepted@date{#1}}
\define@key{@elspaperhead@keys}{available}{\def\@available@date{#1}}
\define@key{@elspaperhead@keys}{keywords}{\def\@keywords{#1}}
\NewEnviron{elspaperhead}[1][]{
  \setkeys{@elspaperhead@keys}{received=,#1}
  \setkeys{@elspaperhead@keys}{revised=,#1}
  \setkeys{@elspaperhead@keys}{accepted=,#1}
  \setkeys{@elspaperhead@keys}{available=,#1}
  \setkeys{@elspaperhead@keys}{keywords=,#1}
  
  \twocolumn[
  \begin{pykzpicture}{latex_common/PykZ}

    \def\@vshift{4mm}
    \def\@logo@height{2cm}

    \tikzset{
      separator/.append style={
        line width=0.1mm
      },
      fatbar/.append style={
        line width=1mm,
        line cap=rect
      },
      review/.append style={
        scale=1.2,
        anchor=north west
      }
    }
    
    \@node{
      anchor=north west,
      yshift=-\@vshift
    } (elslogo) at (0,0)
    {\includegraphics[height=\@logo@height]{latex_common/media/elsevier_logo.pdf}};

    \@node{
      anchor=south east
    } (arcfp) at (\textwidth,-\@logo@height-\@vshift)
    {\includegraphics[height=\@logo@height+\@vshift]{latex_common/media/arc_fp.pdf}};

    \path let \p1=(arcfp.east), \p2=(arcfp.west) in \pgfextra{
      \tikzmath{\@H = \textwidth-(\x1-\x2)-\@vshift;}
      \dimensionalize{\@H}
      \@draw{separator} (0,0) -- ++(\@H,0) node(tmp){};
    };

    \path let \p1=(elslogo.south east), \p2=(tmp) in \pgfextra{
      \@draw{
        fill=black!20,
        draw=none} (\x1+\@vshift,\y1) rectangle (\x2,\y2-\@vshift);
      \@node{scale=1.7} at ({(\x1+\@vshift+\x2)/2},{(\y1+\y2-\@vshift)/2})
      {Annual Reviews in Control};
    };
    
    \@draw{fatbar} ($(elslogo.south west)+(0,-\@vshift)$) node(tmp){} -- ++(\textwidth,0);
    \@node{review} at ($(tmp)+(0,-5mm)$) {Review};
  \end{pykzpicture}
  
  \vspace{5mm}
  \thispagestyle{empty}
  {\LARGE\@elstitle}\medskip

  \xdef\@counter{1}
  \foreach \authorfullnameaffil in  \elsauthorlist {%
    \list@get[affiliation]{\authorfullnameaffil}{1}%
    \list@get[authorfullname]{\authorfullnameaffil}{2}%
    \large%
    {\color{black!50}\elsgetlistelement{\authorfullname}{1}}~% % first name
    \elsgetlistelement{\authorfullname}{2}$^{\text{
        \hyperlink{affil\affiliation}{\affiliation}
      }}$% % last name
    \pgfmathparse{\@counter<\elsauthorcount-1 ? 1:0}%
    \ifthenelse{\pgfmathresult>0}{,~}{}%
    \pgfmathparse{\@counter==\elsauthorcount-1 ? 1:0}%
    \ifthenelse{\pgfmathresult>0}{~and~}{}%
    \elsaddtocounter{\@counter}%
  }
  \vspace{3mm}

  \foreach \affiliation@details in \elsaffiliationlist {%
    \list@get[@letter]{\affiliation@details}{1}
    \list@get[@text]{\affiliation@details}{2}
    {\footnotesize\hypertarget{affil\@letter}{$^{\text{\@letter}}$}~\textit{\@text}}
  }
  \vspace{5mm}
  
  \def\@vsep@big{0.7cm}
  \def\@vsep@small{0.6*\@vsep@big}
  \def\@kw@width{0.25\textwidth}
  \def\@hsep{0.1\textwidth}

  \tikzmath{\@abstract@width=(\textwidth-\@kw@width-\@hsep)/0.9;}
  \dimensionalize{\@abstract@width}

  \newsavebox\Abstract
  \begin{lrbox}{\Abstract}
    \begin{minipage}{\@abstract@width}
      \BODY
    \end{minipage}
  \end{lrbox}

  \begin{pykzpicture}{latex_common/PykZ}

    \tikzset{
      separator/.append style={
        line width=0.1mm
      },
      title/.append style={
        anchor=north west,
        scale=0.9
      },
      info text/.append style={
        scale=0.8,
        anchor=north west,
        yshift=-1mm
      },
      abstract/.append style={
        anchor=north west,
        align=left,
        yshift=-1mm,
        scale=0.9
      }
    }

    \@draw{separator} (0,0) -- ++(\textwidth,0);

    \@node{title} (info) at (0,-\@vsep@big) {A~R~T~I~C~L~E~~~~I~N~F~O};

    \@draw{separator}
    ($(info.south west)+(0,-\@vsep@small)$) --
    ++(\@kw@width,0);

    \@node{info text} (history) at ($(info.south west)+(0,-\@vsep@small)$)
    {\textit{Article history:}};
    \@node{info text} (received) at (history.south west) {Received
      \@receive@date};
    \@node{info text} (revised) at (received.south west) {Revised
      \@revised@date};
    \@node{info text} (accepted) at (revised.south west) {Accepted
      \@accepted@date};
    \@node{info text} (available) at (accepted.south west) {Available
      \@available@date};

    \@draw{separator} ($(available.south west)+(0,-\@vsep@small)$) --
    ++(\@kw@width,0);
    \@node{info text} (keywords_prev) at ($(available.south west)+(0,-\@vsep@small)$)
    {\textit{Keywords:}};

    \foreach \@key in \@keywords {
      \@node{info text} (keywords_prev) at (keywords_prev.south west) {\@key};
    }

    \@node{title} (abstract)
    at (\@kw@width+\@hsep,-\@vsep@big) {A~B~S~T~R~A~C~T};
    
    \@draw{separator} ($(info.south west)+(\@kw@width+\@hsep,-\@vsep@small)$) --
    ++(\textwidth-\@kw@width-\@hsep,0);

    \@node{abstract} (abstract) at ($(info.south west)+(\@kw@width+\@hsep,-\@vsep@small)$)
    {\usebox\Abstract};

    \@node{scale=0.9,anchor=north east,yshift=-1mm}
    (copyright)
    at (abstract.south east)
    {{\small\copyright}~ 2020 Elsevier Ltd. All rights reserved.};

    \@draw{separator} ($(copyright.north east)+(0,-\@vsep@big)$)
    -- ++(-\textwidth,0);
    
  \end{pykzpicture}

  \renewcommand*\l@section{\@dottedtocline{3}{0.5em}{2.3em}}
  \renewcommand\@dotsep{3}
  \renewcommand\tableofcontents{%
    \titleformat{\section}{\normalfont\bfseries\normalsize}{}{0cm}{Contents}
    \section*{\contentsname
      \@mkboth{%
        \MakeUppercase\contentsname}{\MakeUppercase\contentsname}}%
    \@starttoc{toc}%
  }
  % \renewcommand\cftchapafterpnum{\vskip10pt}
  \renewcommand\cftsecafterpnum{\vskip3pt}
  
  \tableofcontents

  \vspace{7mm}
  ]
}
\newcommand{\@node}[1]{\node[#1]}
\newcommand{\@draw}[1]{\draw[#1]}
\makeatother

%%%%%%%%%%%%%%%%%%%%%%%
% Bibliography
%%%%%%%%%%%%%%%%%%%%%%%

\def\elscitecolor{blue!60}
\setcitestyle{authoryear,semicolon}% ,open={((},close={))}}
\hypersetup{
  colorlinks,
  linkcolor={red},
  citecolor={\elscitecolor},
  urlcolor={blue}
}
\renewcommand*{\bibfont}{\footnotesize\color{\elscitecolor}}
\setlength{\bibsep}{1pt}

%%%%%%%%%%%%%%%%%%%%%%%
% Theorems
%%%%%%%%%%%%%%%%%%%%%%%

\newtheorem{theorem}{Theorem}
\newtheorem{corollary}{Corollary}
\newtheorem{lemma}{Lemma}
\newtheorem{remark}{Remark}

\theoremstyle{definition}

\newtheorem{definition}{Definition}
\newtheorem{condition}{Condition}
\newtheorem{assumption}{Assumption}
\newtheorem{property}{Property}
\newtheorem{example}{Example}

%%% Local Variables:
%%% mode: latex
%%% TeX-master: t
%%% End:
