\def\emptyCaption{{\color{white}?}}

\def\optiPadding{1mm}
\def\captionPadding{1mm}

\makeatletter
\newcommand{\clonelabel}[2]{\@bsphack
  \expandafter\ifx\csname r@#2\endcsname\relax
  \else\protected@write\@auxout{}{\string\newlabel{#1}%
    {\csname r@#2\endcsname}}%
  \fi
  \expandafter\ifx\csname r@#2@cref\endcsname\relax
  \else\protected@write\@auxout{}{\string\newlabel{#1@cref}%
    {\csname r@#2@cref\endcsname}}%
  \fi
  \@esphack}
\makeatother

\newcounter{ProblemCounter}
\newcounter{EqStartCounter}
\newcounter{ll}
\newcounter{jj}
\newcounter{kk}

\setcounter{ProblemCounter}{0}
\setcounter{EqStartCounter}{0}

\newcommand{\setBound}[2][true]{%
  % #1: apply \tikzmark if true
  % #2: tikzmark label
  \ifthenelse{\equal{#1}{true}}{
    \tikzmark{#2\theProblemCounter}
  }{}
}
\newcommand{\getBound}[1]{pic cs:#1\theProblemCounter}

\newlength{\lineHeight}

\newcommand{\theBackground}[5]{%
  % #1: \border
  % #2: type {cost,constraint}
  % #3: cost OR constraint list
  % #4: constraint number (only relevant if type=constraint)
  % #5: \pcaption
  \begin{tikzpicture}[remember picture,overlay]%
    % Styles
    \ifthenelse{\equal{#1}{true}}
    {\def\opacityValue{1}}
    {\def\opacityValue{0}}
    \tikzset{
      background/.append style={
        rectangle,
        anchor=south west,
        fill=yellow!20,
        inner sep=0,
        outer sep=0,
        minimum width=\problemWidth+2*\optiPadding,
        opacity=\opacityValue
      },
      outline box/.append style={
        anchor=south west,
        inner sep=0,
        outer sep=0
      },
      border/.append style={
        solid,
        draw=orange!80,
        line width=0.2mm,
        opacity=\opacityValue
      },
      border cap/.append style={
        background,
        border,
        anchor=west,
        minimum height=2*\optiPadding,
        rounded corners=\optiPadding,
        opacity=\opacityValue
      }
    }
    % Cost background
    \ifthenelse{\equal{#2}{cost}}{
      \phantomoutline{#3}
      \edef\costBelowShift{\belowShift}
      \path let
      \p1=(\getBound{cost.south west}),
      \p2=(\getBound{variables.south west})
      in \pgfextra{
        \node[
        background,
        minimum height=\totalHeight
        ] (cost) at (\x1-\optiPadding,\y1-\belowShift) {};
        \draw[border] (cost.south west) to (cost.north west);
        \draw[border] (cost.south east) to (cost.north east);
      };
      % Draw the top cap
      \begin{scope}
        \clip
        ($(cost.north west)+(-\optiPadding,0)$)
        rectangle
        ($(cost.north east)+(2*\optiPadding,2*\optiPadding)$);
        \node[border cap] at (cost.north west) {};
      \end{scope}
      \xdef\lastLine{cost}
      % Caption
      \ifthenelse{\equal{\writeCaption}{true}}{
        \node[
        anchor=south east,
        align=right,
        inner sep=0,
        outer sep=0,
        font={\bfseries\footnotesize\baselineskip=1.3em}
        ] (problemLabel) at ($(cost.south west)+
        (-\captionPadding,\costBelowShift*1pt)$)
        {\ifthenelse{\equal{\ptag}{}}{Problem \theequation.}{Problem \ptag.}};
        \node[
        anchor=north east,
        align=right,
        inner sep=0,
        outer sep=0,
        font={\itshape\footnotesize\baselineskip=1.3em}
        ] (description) at ($(problemLabel.south east)+(0,-2*\captionPadding)$)
        {#5};
      }{}
    }{}
    % Constraint background
    \ifthenelse{\equal{#2}{constraint}}{
      \coordinate (previous) at (\lastLine.south west);
      \phantomoutline{#3[#4]}
      \path let
      \p1 = (\getBound{constraint#4.south west}),
      \p2 = (previous)
      in \pgfextra{
        \tikzset{
          marker/.append style={
            inner sep=0,
            outer sep=0,
            minimum size=1mm,
            fill=black,
            circle
          }
        }
        \ifthenelse{\equal{#4}{\listlen#3[]}}{\tikzmath{\padExtraBelow=0;}}%
        {\tikzmath{\padExtraBelow=\optiPadding;}}
        \pgfmathparse{\y1<\y2 ? 1 : 0}
        \ifthenelse{\pgfmathresult>0}{
          \tikzmath{\lineHeight=\y2-(\y1-\belowShift)+\padExtraBelow;}
        }{
          \tikzmath{\lineHeight=\totalHeight+2*\padExtraBelow;}
        }
        \tikzmath{\constraintBgShiftX=\costResultWidth+\optiPadding;}
        \node[
        background,
        minimum height=\lineHeight
        ] (constraint#4) at (\x1-\constraintBgShiftX,\y1-\belowShift-\padExtraBelow) {};
        \draw[border] (constraint#4.south west) to (constraint#4.north west);
        \draw[border] (constraint#4.south east) to (constraint#4.north east);
      };
      \xdef\lastLine{constraint#4}
      % Draw the bottom cap
      \ifthenelse{\equal{#4}{\listlen#3[]}}{
        \begin{scope}
          \clip
          ($(\lastLine.south west)+(-\optiPadding,0)$)
          rectangle
          ($(\lastLine.south east)+(2*\optiPadding,-2*\optiPadding)$);
          \node[border cap] at (\lastLine.south west) {};
        \end{scope}
      }{}
    }{}
  \end{tikzpicture}%
}

\newcommand{\printCost}[6][true]{
  % #1: print with \setBound and alignment if true
  % #2: \result
  % #3: \constrained
  % #4: \task
  % #5: \variables
  % #6: \objective
  \setBound[#1]{cost.south west}%
  \ifthenelse{\equal{#2}{}}{}{#2=}%
  \setBound[#1]{cost.south west2}%
  \ifthenelse{\equal{#3}{true}}{\ifthenelse{\equal{#1}{true}}{&}{}}{}%
  #4_{\setBound[#1]{variables.south west}#5\setBound[#1]{bottom}}#6%
  \setBound[#1]{cost.south east}%
  \ifthenelse{\equal{#3}{true}}{%
    ~\mathrm{s.t.}\setBound[#1]{cost.south east}
  }{}%
  % determine width of the "result =" part
  \begin{tikzpicture}[remember picture,overlay]%
    \path let
    \p1=(\getBound{cost.south west}),
    \p2=(\getBound{cost.south west2})
    in \pgfextra{
      \tikzmath{\costResultWidth=\x2-\x1;}
      \xdef\costResultWidth{\costResultWidth}
    };
  \end{tikzpicture}%
}

\newcommand{\OPTBG}[4]{%
  \theBackground%
  {\border}%
  {#1}%
  {#2}%
  {#3}%
  {#4}
}

\newcommand{\theProblem}[8]{
  % #1: \result
  % #2: \constrained
  % #3: \task
  % #4: \variables
  % #5: \objective
  % #6: list of constraints
  % #7: \plabel
  % #8: \pcaption
  % Print the optimization problem
  \OPTBG{cost}{\printCost[false]{#1}{#2}{#3}{#4}{#5}}{0}{#8}%
  \printCost[true]{#1}{#2}{#3}{#4}{#5}%
  \ifthenelse{\equal{#7}{}}{}{\label{eq:#7_a}}
  \ifthenelse{\equal{#2}{true}}{%
    \\%
    \forloop{kk}{0}{\arabic{kk}<\listlen#6[]}{%
      \setcounter{jj}{\value{kk}+1}%
      \setcounter{ll}{\value{kk}+2}%
      &\setBound{constraint\arabic{jj}.south west}%
      \OPTBG{constraint}{#6}{\arabic{jj}}{}%
      #6[\arabic{jj}]%
      \setBound{constraint\arabic{jj}.south east}%
      \ifthenelse{\equal{#7}{}}{}{\label{eq:#7_\alph{ll}}}%
      \ifthenelse{\equal{\arabic{jj}}{\listlen#6[]}}{}{\\}
    }
  }{}
}

\newcommand{\phantomoutline}[1]{
  % \node[outline box] (outline) at (\getBound{cost.south west})
  % {\phantom{$\displaystyle#1$}};
  \node[
  draw=none,
  inner sep=0
  ] (outline) at (0,0)
  {\pgfmark{eqstart\theEqStartCounter}\phantom{$\displaystyle#1$}};
  % \node[
  % draw=black,
  % inner sep=0
  % ] (outline) at (5,0) {\pgfmark{eqstart\theEqStartCounter}{$\displaystyle#1$}};
  \path let
  \p1=(outline.north east),
  \p2=(outline.south east),
  \p3=(pic cs:eqstart\theEqStartCounter)
  in \pgfextra{
    \tikzmath{
      \totalHeight=\y1-\y2;
      \belowShift=\y3-\y2;}
    \xdef\totalHeight{\totalHeight}
    \xdef\belowShift{\belowShift}
  };
  \stepcounter{EqStartCounter}
}

\newcommand{\computeProblemWidth}[2]{
  % #1: \constrained
  % #2: list of constraints
  % Compute the problem width
  \begin{tikzpicture}[remember picture,overlay]
    \path let
    \p1=(\getBound{cost.south west}),
    \p2=(\getBound{cost.south east})
    in \pgfextra{
      \tikzmath{\problemWidth=\x2-\x1;}
      \xdef\problemWidth{\problemWidth}
    };
    \ifthenelse{\equal{#1}{true}}{
      \foreach \i in {1,...,\listlen#2[]} {
        \path let
        \p1=(\getBound{cost.south west}),
        \p2=(\getBound{constraint\i.south east})
        in \pgfextra{
          \tikzmath{\problemWidth=max(\problemWidth,\x2-\x1);}
          \xdef\problemWidth{\problemWidth}
        };
      }
    }{}
  \end{tikzpicture}
}

\makeatletter
\define@key{optimusKeys}{task}{\def\task{#1}}%
\define@key{optimusKeys}{objective}{\def\objective{#1}}%
\define@key{optimusKeys}{variables}{\def\variables{#1}}%
\define@key{optimusKeys}{result}{\def\result{#1}}%
\define@key{optimusKeys}{plabel}{\def\plabel{#1}}%
\define@key{optimusKeys}{ptag}{\def\ptag{#1}}%
\define@key{optimusKeys}{border}{\def\border{#1}}%
\define@key{optimusKeys}{pcaption}{\def\pcaption{#1}}%
\makeatother

\NewEnviron{optimus}[1][]{%
  % Parameters
  \setkeys{optimusKeys}{plabel=,#1}%
  \setkeys{optimusKeys}{border=false,#1}%
  \setkeys{optimusKeys}{task=\min,#1}%
  \setkeys{optimusKeys}{objective=,#1}%
  \setkeys{optimusKeys}{variables=,#1}%
  \setkeys{optimusKeys}{result=,#1}%
  \setkeys{optimusKeys}{ptag=,#1}%
  \setkeys{optimusKeys}{pcaption=,#1}
  %
  % Prepare variables
  \stepcounter{ProblemCounter}%
  \ifthenelse{\equal{\BODY}{}}{%
    \xdef\constrained{false}}{%
    \xdef\constrained{true}
    \setsepchar{\#}%
    \readlist\constraints{\BODY}}
  \ifthenelse{\equal{\pcaption}{}}{
    \xdef\writeCaption{false}}{
    \xdef\writeCaption{true}}
  % Print the problem
  \computeProblemWidth{\constrained}{\constraints}%
  \newcommand{\printProblemBody}{
    \theProblem%
    {\result}%
    {\constrained}%
    {\task}%
    {\variables}%
    {\objective}%
    {\constraints}%
    {\plabel}%
    {\pcaption}
  }
  \ifthenelse{\equal{\constrained}{true}}{
    \ifthenelse{\equal{\plabel}{}}{
      \begin{align*}
        \printProblemBody
      \end{align*}
    }{
      \ifthenelse{\equal{\ptag}{}}{
        \begin{subequations}
          \label{problem:\plabel}
          \begin{align}
            \printProblemBody
          \end{align}
        \end{subequations}
      }{
        \begin{taggedsubequations}{\ptag}
          \label{problem:\plabel}
          \begin{align}
            \printProblemBody
          \end{align}
        \end{taggedsubequations}
      }
    }
  }{
    \ifthenelse{\equal{\plabel}{}}{
      \begin{equation*}
        \printProblemBody
      \end{equation*}
    }{
      \begin{equation}
        \printProblemBody
      \end{equation}\clonelabel{problem:\plabel}{eq:\plabel_a}
    }
  }
}

%%% Local Variables:
%%% mode: latex
%%% TeX-master: t
%%% End:
