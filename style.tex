% No paragraph indentation
\setlength\parindent{0pt}

% Paragraph spacing
\setlength{\parskip}{4pt}

% List spacing
% \setlist{itemsep=0pt}

% Hyperlinks
\hypersetup{
  colorlinks,
  linkcolor={red},
  citecolor={green},
  urlcolor={blue}
}

\setcounter{MaxMatrixCols}{20}

% Units
\DeclareSIUnit\century{century}
\DeclareSIUnit\year{yr}
\DeclareSIUnit\pound{lb}
\DeclareSIUnit\mile{mi}
\DeclareSIUnit\byte{B}

% Equation box style
\tcbset{highlight math style={enhanced,
colframe=red!60!black,colback=yellow!50!white,arc=4pt,boxrule=1pt,
drop fuzzy shadow}}

% Table column types
\renewcommand\tabularxcolumn[1]{m{#1}} % Vertical centering
\newcolumntype{C}{>{\centering\arraybackslash}X}
\newcolumntype{L}{>{\raggedright\arraybackslash}X}
\newcolumntype{R}{>{\raggedleft\arraybackslash}X}

% Theorem environments
\newtheorem{theorem}{Theorem}
\newtheorem{lemma}{Lemma}
\newtheorem{corollary}{Corollary}
\newtheorem{proposition}{Proposition}

\theoremstyle{definition}

\newtheorem{remark}{Remark}
\newtheorem{property}{Property}
\newtheorem{condition}{Condition}
\newtheorem{method}{Method}
\newtheorem{example}{Example}
\newtheorem{problem}{Problem}
\newtheorem{definition}{Definition}
\newtheorem{assumption}{Assumption}

\theoremstyle{empty}

\newtheorem{refproof}{Proof}

\declaretheoremstyle[
  headfont=\color{black}\normalfont\bfseries,
  bodyfont=\color{darkgray}\normalfont,
]{colored}

\declaretheorem[style=colored,name=Definition]{definitionc}
% \AtEndEnvironment{definitionc}{\null\hfill\qedsymbol}%
\declaretheorem[style=colored,name=Condition]{conditionc}
\declaretheorem[style=colored,name=Assumption]{assumptionc}
\declaretheorem[style=colored,name=Remark]{remarkc}
\declaretheorem[style=colored,name=Theorem]{theoremc}
% \AtEndEnvironment{definitionc}{\null\hfill\qedsymbol}%

% Boxes
\newmdenv[
outerlinewidth = 1,%
linewidth = 0pt,%
roundcorner = 2pt,%
leftmargin = 20,%
rightmargin = 0,%
backgroundcolor = lightgray!10,%
outerlinecolor = gray!50,%
innertopmargin = \topskip,%
splittopskip = \topskip,%
frametitle = Summary,%
frametitlebelowskip = 0pt,%
]{summary_box}

\newmdenv[
outerlinewidth = 1,%
linewidth = 0pt,%
roundcorner = 2pt,%
leftmargin = 60,%
rightmargin = 60,%
backgroundcolor = yellow!40,%
outerlinecolor = black!50,%
innertopmargin = \topskip,%
splittopskip = \topskip,%
]{highlight_box}

% Nomenclature

\newif\ifnomentry
\renewcommand{\nomlabel}[1]{#1\hfill\hspace{\labelsep}\ifnomentry$=$\fi}
\newlength{\nomwidest}
\renewcommand{\nomgroup}[1]{%
\nomentryfalse
\ifthenelse{\equal{#1}{V}}{\item[\textbf{Variables}]}{%
\ifthenelse{\equal{#1}{A}}{\item[\textbf{Abbreviations}]}{%
\ifthenelse{\equal{#1}{E}}{\item[\textbf{Equations}]}{}}}
\nomentrytrue
}
\setlength{\nomitemsep}{-\parsep} 
\newcommand{\defvar}[3][show]{\nomenclature[V]{#2}{#3}\ifthenelse{\equal{#1}{show}}{#2}{}}
\newcommand{\defabbrev}[3][show]{\nomenclature[A]{#2}{#3}\ifthenelse{\equal{#1}{show}}{#3 (#2)}{}}
\newcommand{\defeq}[2]{\nomenclature[E]{#1}{#2}}

% \setlength{\nomlabelwidth}{3cm}
% \renewcommand{\nomlabel}[1]{#1\hfill\hspace{\labelsep}\ifnomentry$=$\fi}
% \newlength{\nomwidest}
% \renewcommand{\nomgroup}[1]{%
% \nomentryfalse
% \ifthenelse{\equal{#1}{V}}{\item[\textbf{Variables}]}{%
% \ifthenelse{\equal{#1}{A}}{\item[\textbf{Abbreviations}]}{%
% \ifthenelse{\equal{#1}{E}}{\item[\textbf{Equations}]}{}}}
% \nomentrytrue
% }
% \setlength{\nomitemsep}{-\parsep} 
% \newcommand{\defvar}[3][show]{\nomenclature[V]{#2}{#3}\ifthenelse{\equal{#1}{show}}{#2}{}}
% \newcommand{\defabbrev}[3][show]{\nomenclature[A]{#2}{#3}\ifthenelse{\equal{#1}{show}}{#3 (#2)}{}}
% \newcommand{\defeq}[2]{\nomenclature[E]{#1}{#2}}

% Algorithms

\algrenewcommand\algorithmicindent{1em}
\algrenewcommand\alglinenumber[1]{{\sf\footnotesize#1}}
\algtext*{EndWhile}% Remove "end while" text
\algtext*{EndIf}% Remove "end if" text
\algtext*{EndFor}% Remove "end for" text

\algrenewcommand\algorithmicrequire{\textbf{Precondition:}}
\algrenewcommand\algorithmicensure{\textbf{Postcondition:}}
\algnewcommand\algorithmicinput{\textbf{Input:}}
\algnewcommand\Input{\item[\algorithmicinput]}
\algnewcommand\algorithmicfunctions{\textbf{Lambda Functions:}}
\algnewcommand\Functions{\item[\algorithmicfunctions]}
\algnewcommand\Stop[1][]{\State \textbf{STOP} {\color{green} \ul{#1}}}
\algnewcommand\Break{\State \textbf{break}}
\algnewcommand\Continue{\State \textbf{continue}\xspace}
\algnewcommand\Error[1][]{\State \textbf{error} {\color{red} \uwave{#1}}}
\algrenewcommand\algorithmiccomment[1]{\hfill {\color{gray} \(\triangleright\) #1}}
\makeatletter
\algnewcommand\Section[1]{\Statex \hskip\ALG@thistlm \textbf{\color{gray} \(\triangleright\) #1}}
\makeatother
\algdef{SE}[PARAMETERS]{Parameters}{EndParameters}
   {\algorithmicparameters}
   {\algorithmicend\ \algorithmicparameters}
\algnewcommand{\algorithmicparameters}{\textbf{parameters}}
\newcommand{\algorithmiclambda}[3][]{\ifthenelse{\equal{#1}{}}{\texttt{#2}(#3)}{$\texttt{#2}\gets\textbf{lambda}~#3:~#1$}}

% Allow breaking long equations across pages
\allowdisplaybreaks

%%% Local Variables:
%%% mode: plain-tex
%%% TeX-master: t
%%% End:
