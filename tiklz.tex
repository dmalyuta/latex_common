%%%%%%%%%%%% Packages

\usepackage{pgfplots}
\pgfplotsset{compat=newest}

%%%%%%%%%%%% Libraries

\usetikzlibrary{math}
\usetikzlibrary{calc}
\usetikzlibrary{shapes}
\usetikzlibrary{arrows}
\usetikzlibrary{tikzmark}
\usetikzlibrary{positioning}
\usetikzlibrary{scopes}
\usetikzlibrary{shapes.multipart}
\usetikzlibrary{shapes.geometric}
\usetikzlibrary{bending}
\usetikzlibrary{decorations.markings}
\usetikzlibrary{intersections}

% curve through points: (A) to[curve through={(B) .. (C) .. (...)}] {Z};
% \usetikzlibrary{hobby}

%%%%%%%%%%%% Code Extensions

%%%%% Union of shapes
% Usage: \fill[draw on back,...] ...
% Source: https://tex.stackexchange.com/a/11620/90535

\pgfdeclarelayer{back}
\pgfsetlayers{back,main}

\tikzset{%
  on layer/.code={
    \pgfonlayer{#1}\begingroup
    \aftergroup\endpgfonlayer
    \aftergroup\endgroup
  },%
  draw on back/.style={
    preaction={
      draw,
      on layer=back,
      line width=#1
    }
  }
}

%%%%%%%%%%%% Custom shapes

\newcommand{\tiklzrectangle}[5]
{
  % Draw a rectangle
  %
  % #1: what drawing command to execute
  % #2: rectangle center x position
  % #3: rectangle center y position
  % #4: rectangle width
  % #5: rectangle height
  \tikzmath
  {
    \x=#2;
    \y=#3;
    \w=#4;
    \h=#5;
  }
  #1 (\x-\w/2,\y-\h/2) rectangle (\x+\w/2,\y+\h/2);
}

\newcommand{\tiklztriangle}[4]
{
  % Draw an isosceles triangle
  %
  % #1: what drawing command to execute
  % #2: triangle center x position
  % #2: triangle center y position
  % #2: triangle sidelength
  \tikzmath
  {
    \x=#2;
    \y=#3;
    \r=#4;
  }
  #1 (\x-\r/2,{\y-\r/2/sqrt(3)}) --
  (\x+\r/2,{\y-\r/2/sqrt(3)}) --
  (\x,{\y+\r/sqrt(3)}) --
  cycle;
}

\newcommand{\tiklzparabola}[5]
{
  % Draw a parabola
  %
  % #1: t0, the left x-coordinate
  % #2: t1, the right x-coordinate
  % #3: y0, the left y-coordinate
  % #4: y1, the right y-coordinate
  % #5: theta0, the parabola trajectory angle on the left
  \tikzmath
  {
    \t0=#1;
    \t1=#2;
    \y0=#3;
    \y1=#4;
    \myangle=#5;
    \a=(\y1-\y0-tan(\myangle)*(\t1-\t0))/(\t1*\t1-\t0*\t0-2*\t0*(\t1-\t0));
    \b=tan(\myangle)-2*\a*\t0;
    \c=\y0-\a*\t0*\t0-\b*\t0;
  }
}

%%%%%%%%%%%% Helper grid

\makeatletter
\def\grd@save@target#1{%
  \def\grd@target{#1}}
\def\grd@save@start#1{%
  \def\grd@start{#1}}
\def\grdOpacity{0.5}
\tikzset{
  grid with coordinates/.style={
    to path={%
      \pgfextra{%
        \edef\grd@@target{(\tikztotarget)}%
        \tikz@scan@one@point\grd@save@target\grd@@target\relax
        \edef\grd@@start{(\tikztostart)}%
        \tikz@scan@one@point\grd@save@start\grd@@start\relax
        \draw[minor help lines] (\tikztostart) grid (\tikztotarget);
        \draw[major help lines] (\tikztostart) grid (\tikztotarget);
        \grd@start
        \pgfmathsetmacro{\grd@xa}{\the\pgf@x/1cm}
        \pgfmathsetmacro{\grd@ya}{\the\pgf@y/1cm}
        \grd@target
        \pgfmathsetmacro{\grd@xb}{\the\pgf@x/1cm}
        \pgfmathsetmacro{\grd@yb}{\the\pgf@y/1cm}
        \pgfmathsetmacro{\grd@xc}{\grd@xa + \pgfkeysvalueof{/tikz/grid with coordinates/major step}}
        \pgfmathsetmacro{\grd@yc}{\grd@ya + \pgfkeysvalueof{/tikz/grid with coordinates/major step}}
        \foreach \x in {\grd@xa,\grd@xc,...,\grd@xb}
        \node[anchor=north] at (\x,\grd@ya) {\pgfmathprintnumber{\x}};
        \foreach \y in {\grd@ya,\grd@yc,...,\grd@yb}
        \node[anchor=east] at (\grd@xa,\y) {\pgfmathprintnumber{\y}};
      }
    }
  },
  minor help lines/.style={
    help lines,
    opacity=\grdOpacity,
    step=\pgfkeysvalueof{/tikz/grid with coordinates/minor step}
  },
  major help lines/.style={
    help lines,
    opacity=\grdOpacity,
    line width=\pgfkeysvalueof{/tikz/grid with coordinates/major line width},
    step=\pgfkeysvalueof{/tikz/grid with coordinates/major step}
  },
  grid with coordinates/.cd,
  minor step/.initial=.2,
  major step/.initial=1,
  major line width/.initial=1pt
}

\newcommand{\tiklzhelpgrid}{\draw (-7,-4) to[grid with coordinates] (7,4);}
