%%%%%%%%%%%% Packages

\usepackage{pgfplots}
\pgfplotsset{compat=newest}

%%%%%%%%%%%% Libraries

\usetikzlibrary{math}
\usetikzlibrary{calc}
\usetikzlibrary{shapes}
\usetikzlibrary{arrows}
\usetikzlibrary{tikzmark}
\usetikzlibrary{positioning}
\usetikzlibrary{scopes}
\usetikzlibrary{shapes.multipart}
\usetikzlibrary{shapes.geometric}
\usetikzlibrary{bending}
\usetikzlibrary{decorations.markings}
\usetikzlibrary{intersections}

% curve through points: (A) to[curve through={(B) .. (C) .. (...)}] {Z};
% \usetikzlibrary{hobby}

%%%%%%%%%%%% Common settings

\newcommand{\tiklzcommonsettings}{
  \tikzset{
    every node/.append style={
      inner sep=0,
      outer sep=0
    },
    axis/.append style={
      black,
      ->
    },
    label text/.append style={
      outer sep=1
    }
  }
}

%%%%%%%%%%%% Cute clock

\newcommand{\cuteclock}[1][]{
  \begin{scope}[#1]
    \tikzset{
      skinny/.append style={
        inner sep=0,
        outer sep=0
      },
      clockHand/.append style={
        line width=0.2mm
      },
      clockFoot/.append style={
        line width=0.3mm
      }
    }
    
    \tikzmath{
      \R = 3mm;
      \r = 0.3mm;
      \angleHour = 90;
      \angleMinute = -45;
      \lengthHour = 0.5*\R;
      \lengthMinute = 0.9*\R;
      \earSize = 1mm;
      \angleEar = 45;
      \shiftEar = 1.1*\R;
      \lengthFoot = 1mm;
      \angleFoot= 30;
    }
  
    \def\rimColor{black}
    \def\faceColor{white}

    \node[skinny,circle,draw=\rimColor,fill=\faceColor,
    minimum size=2*\R] (cuteclock) at (0,0) {};
  
    \draw[clockHand,rotate=\angleHour] (0,0) to (\lengthHour*1pt,0);
    \draw[clockHand,rotate=\angleMinute] (0,0) to (\lengthMinute*1pt,0);

    \draw[\rimColor,fill=\rimColor] (\r*1pt,0) arc (0:360:\r*1pt);
  
    \foreach \earAngle in {-\angleEar,\angleEar} {
      \draw[\rimColor,fill=\faceColor,rotate=\earAngle,shift={(0,\shiftEar*1pt)}]
      (-\earSize*1pt,0) to (\earSize*1pt,0) arc (0:180:\earSize*1pt) to cycle;
    }
  
    \foreach \footAngle in {-\angleFoot,\angleFoot} {
      \draw[clockFoot,
      rotate=\footAngle,shift={(0,-\R*1pt)}] (0,0) to (0,-\lengthFoot*1pt);
    }
  \end{scope}
}

%%%%%%%%%%%% Arrow as a bullet point

\newcommand{\fancyarrow}{
\resizebox{!}{0.7em}{
\begin{tikzpicture}
  \tikzmath{
    \width=0.7em;
    \height=0.5em;
    \inset=0.2*\width;
  }
  \draw[line width=2mm] (0,0) to (-\inset,\height/2)
  to (-\inset+\width,0) to
  (-\inset,-\height/2) to cycle;
  \path[fill=black] (0,0) to (-\inset,\height/2) to
  (-\inset+\width,0) to cycle;
\end{tikzpicture}
}
}

%%%%%%%%%%%% Tentacle for "mind-map" style connections

\makeatletter
\define@key{tentaclekeys}{start}{\def\xys{#1}}%
\define@key{tentaclekeys}{end}{\def\xye{#1}}%
\define@key{tentaclekeys}{out}{\def\out{#1}}%
\define@key{tentaclekeys}{in}{\def\in{#1}}%
\define@key{tentaclekeys}{widthstart}{\def\ws{#1}}%
\define@key{tentaclekeys}{widthend}{\def\we{#1}}%
\define@key{tentaclekeys}{color}{\def\color{#1}}%
\makeatother
\NewEnviron{tentacle}[1][]{
  % Parameter loading
  \setkeys{tentaclekeys}{start={0,0},#1}
  \setkeys{tentaclekeys}{end={1,-1},#1}
  \setkeys{tentaclekeys}{out=-90,#1}
  \setkeys{tentaclekeys}{in=90,#1}
  \setkeys{tentaclekeys}{widthstart=1mm,#1}
  \setkeys{tentaclekeys}{widthend=0mm,#1}
  \setkeys{tentaclekeys}{color=black,#1}
  % Drawing
  \tikzset{
    tip/.append style={
      color=\color,
      fill=\color,
      line width=0.1mm
    },
    tentacle/.append style={
      color=\color,
      fill=\color,
      line width=0.1mm
    }
  }

  \tikzmath{
    \angss = \out+80;
    \angse = \out+280;
    \anges = \in-100;
    \angee = \in-280;
    \xshfts = cos(\angss)*\ws/2;
    \yshfts = sin(\angss)*\ws/2;
    \xshfte = cos(\anges)*\we/2;
    \yshfte = sin(\anges)*\we/2;
  }
  
  \coordinate (start) at (\xys);
  \coordinate (end) at (\xye);
  \coordinate (start right) at ($(start)+(\xshfts*1pt,\yshfts*1pt)$);
  \coordinate (start left) at ($(start)-(\xshfts*1pt,\yshfts*1pt)$);
  \coordinate (end right) at ($(end)+(\xshfte*1pt,\yshfte*1pt)$);
  \coordinate (end left) at ($(end)-(\xshfte*1pt,\yshfte*1pt)$);
  
  \draw[tip] (start right) arc (\angss:\angse:\ws/2);
  \draw[tip] (end right) arc (\anges:\angee:\we/2);

  \draw[tentacle] (start right) to[out=\out,in=\in] (end right)
  to (end left) to[out=\in,in=\out] (start left);
}

%%%%%%%%%%%% Helper grid

\makeatletter
\def\grd@save@target#1{%
  \def\grd@target{#1}}
\def\grd@save@start#1{%
  \def\grd@start{#1}}
\def\grdOpacity{0.5}
\tikzset{
  grid with coordinates/.style={
    to path={%
      \pgfextra{%
        \edef\grd@@target{(\tikztotarget)}%
        \tikz@scan@one@point\grd@save@target\grd@@target\relax
        \edef\grd@@start{(\tikztostart)}%
        \tikz@scan@one@point\grd@save@start\grd@@start\relax
        \draw[minor help lines] (\tikztostart) grid (\tikztotarget);
        \draw[major help lines] (\tikztostart) grid (\tikztotarget);
        \grd@start
        \pgfmathsetmacro{\grd@xa}{\the\pgf@x/1cm}
        \pgfmathsetmacro{\grd@ya}{\the\pgf@y/1cm}
        \grd@target
        \pgfmathsetmacro{\grd@xb}{\the\pgf@x/1cm}
        \pgfmathsetmacro{\grd@yb}{\the\pgf@y/1cm}
        \pgfmathsetmacro{\grd@xc}{\grd@xa + \pgfkeysvalueof{/tikz/grid with coordinates/major step}}
        \pgfmathsetmacro{\grd@yc}{\grd@ya + \pgfkeysvalueof{/tikz/grid with coordinates/major step}}
        \foreach \x in {\grd@xa,\grd@xc,...,\grd@xb}
        \node[anchor=north] at (\x,\grd@ya) {\pgfmathprintnumber{\x}};
        \foreach \y in {\grd@ya,\grd@yc,...,\grd@yb}
        \node[anchor=east] at (\grd@xa,\y) {\pgfmathprintnumber{\y}};
      }
    }
  },
  minor help lines/.style={
    help lines,
    opacity=\grdOpacity,
    step=\pgfkeysvalueof{/tikz/grid with coordinates/minor step}
  },
  major help lines/.style={
    help lines,
    opacity=\grdOpacity,
    line width=\pgfkeysvalueof{/tikz/grid with coordinates/major line width},
    step=\pgfkeysvalueof{/tikz/grid with coordinates/major step}
  },
  grid with coordinates/.cd,
  minor step/.initial=.2,
  major step/.initial=1,
  major line width/.initial=1pt
}

\newcommand{\tiklzhelpgrid}[1][7,4]{%
\def\axmax{#1}
\xdef\axmin{}
\xdef\count{1}
\foreach \val in \axmax {
\ifthenelse{\equal{\count}{1}}{
\xdef\axmin{-\val}
}{
\xdef\axmin{\axmin,-\val}
}
\xdef\count{2}
}
\draw (\axmin) to[grid with coordinates] (\axmax);
}
